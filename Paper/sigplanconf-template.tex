%-----------------------------------------------------------------------------
%
%               Template for sigplanconf LaTeX Class
%
% Name:         sigplanconf-template.tex
%
% Purpose:      A template for sigplanconf.cls, which is a LaTeX 2e class
%               file for SIGPLAN conference proceedings.
%
% Guide:        Refer to "Author's Guide to the ACM SIGPLAN Class,"
%               sigplanconf-guide.pdf
%
% Author:       Paul C. Anagnostopoulos
%               Windfall Software
%               978 371-2316
%               paul@windfall.com
%
% Created:      15 February 2005
%
%-----------------------------------------------------------------------------


\documentclass[preprint,10pt]{sigplanconf}

% The following \documentclass options may be useful:

% preprint      Remove this option only once the paper is in final form.
% 10pt          To set in 10-point type instead of 9-point.
% 11pt          To set in 11-point type instead of 9-point.
% numbers       To obtain numeric citation style instead of author/year.

\usepackage{amsmath}

\newcommand{\cL}{{\cal L}}

\begin{document}

\special{papersize=8.5in,11in}
\setlength{\pdfpageheight}{\paperheight}
\setlength{\pdfpagewidth}{\paperwidth}

\conferenceinfo{CONF 'yy}{Month d--d, 20yy, City, ST, Country}
\copyrightyear{20yy}
\copyrightdata{978-1-nnnn-nnnn-n/yy/mm}
\copyrightdoi{nnnnnnn.nnnnnnn}

% Uncomment the publication rights you want to use.
%\publicationrights{transferred}
%\publicationrights{licensed}     % this is the default
%\publicationrights{author-pays}

\titlebanner{banner above paper title}        % These are ignored unless
\preprintfooter{short description of paper}   % 'preprint' option specified.

\title{Title Text}
\subtitle{Subtitle Text, if any}

\authorinfo{Name1}
           {Affiliation1}
           {Email1}
\authorinfo{Name2\and Name3}
           {Affiliation2/3}
           {Email2/3}

\maketitle

\begin{abstract}
Studies in the field of programming language design evaluation have shown that there exists a gap between small internal methods and large-scale surveys and evaluation methods. This leaves language designers and especially students developing new programming languages with no low-cost solution for language evaluation. In this report, as a starting point, we examine the applicability of the discount usability method on programming languages by surveying relevant literature. Our findings suggest that it is good for examining the IDE and compiler of a language, but is less suited for examining the language’s design. For this reason, we conducted a usability evaluation experiment on a language without using an IDE or a compiler. This lead to the creation of a new method which used the discount usability method and the IDA method as a basis. The usability evaluation experiment was carried out on Quorum using programmers with experience in C and C\#. Most of the problems found in the evaluation were related to the programmers pre-existing expectations of a language. When comparing our results with Quorum’s data, however, we found several discrepancies.  The results show that our evaluation method could serve as a low-cost way of evaluating programming languages for language designers. 
\end{abstract}

\category{D.3}{Programming Languages}{}
\category{H.5.2}{Information Interfaces and Presentation (e.g., HCI)}{User Interfaces}[User-centered design]

% general terms are not compulsory anymore,
% you may leave them out
\terms
Programming Language Design Evaluation, User Evaluation

\keywords
Quorum, Usability Evaluation, Language Design

\section{Introduction}
Computer programming has increasing relevance to today's advancement of technologies. Therefore, existing and established programming languages are constantly improved and new ones are created to meet that demand. Some languages are considered arguably better than others in their intended purpose in the software industry. However, formal evaluation methods for assessing programming languages are very few and limited in their use and most evidence gathered to support such claims are anecdotal in nature\cite{StakingClaims}. 

In recent years, however, the scientific community has tried to rectify this.
In particular, the focus of the PLATEU conferences is the scientific evaluation of languages.
The basic observation is that language use and preference is highly opinionated, which lead to user-based evaluation being the norm for programming languages.
Commonly, the scientific community has made use of methods from social sciences, which usually requires studying a large number of subjects\cite{SocioPLT}\cite{AliceCS1}\cite{BlockOrNot}\cite{FromScratch}.

However, these methods are rather expensive since conducting a quantitative test requires a large group of people. Typically, this means programming language designers cannot do these tests before the language has already gained widespread use. Instead some designers have decided to use more qualitative and lightweight approaches for language evaluation. A commonly used lightweight approach is the discount usability evaluation method. Some examples of using the method in practice are: The language HANDS developed by Pane et al. designed specifically for children; Koitz and Slany used it to on Scratch and their phone language Pocket Code to compare the two; Faldborg and Pedersen used the method to test their spoken programming language LARM;  Faldborg and Nielsen have used it while conducting an empirical experiment on Dart and web-enabled IDE, developed by them, called DartPad.

The primary problem with the discount usability method in such context, as identified by Faldborg and Pedersen and Faldborg and Nielsen, is the difficulty of separating the feedback about the language design from the IDE.
These observations, along with our own experiences, lead us to believe that the discount usability evaluation method is good when evaluating the full package of a language with its IDE and compiler, but is less suited for evaluating language design.

To rectify this, we wanted to create a new evaluation method which will fill the gap between 
internal language analysis and the big and bulky industry programming evaluation methods\cite{AliceCS1}\cite{BlockOrNot}\cite{FromScratch}. This new method used the discount usability evaluation and the IDA methods as a basis, with the main difference being to avoid the use of a compiler or an IDE. To test this method, we used Quorum, an evidence-based programming language. This entailed conducting a qualitative experiment with six experienced programmers as participants.
We believe that our method would be a valuable, low-cost tool for user evaluation of programming languages. 

\appendix
\section{Appendix Title}

This is the text of the appendix, if you need one.

\acks

Acknowledgments, if needed.

% We recommend abbrvnat bibliography style.

\bibliographystyle{abbrvnat}

% The bibliography should be embedded for final submission.

\begin{thebibliography}{}
\softraggedright

\bibitem[Smith et~al.(2009)Smith, Jones]{smith02}
P. Q. Smith, and X. Y. Jones. ...reference text...

\end{thebibliography}


\end{document}
