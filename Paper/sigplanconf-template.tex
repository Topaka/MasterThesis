%-----------------------------------------------------------------------------
%
%               Template for sigplanconf LaTeX Class
%
% Name:         sigplanconf-template.tex
%
% Purpose:      A template for sigplanconf.cls, which is a LaTeX 2e class
%               file for SIGPLAN conference proceedings.
%
% Guide:        Refer to "Author's Guide to the ACM SIGPLAN Class,"
%               sigplanconf-guide.pdf
%
% Author:       Paul C. Anagnostopoulos
%               Windfall Software
%               978 371-2316
%               paul@windfall.com
%
% Created:      15 February 2005
%
%-----------------------------------------------------------------------------


\documentclass[preprint,10pt]{sigplanconf}

% The following \documentclass options may be useful:

% preprint      Remove this option only once the paper is in final form.
% 10pt          To set in 10-point type instead of 9-point.
% 11pt          To set in 11-point type instead of 9-point.
% numbers       To obtain numeric citation style instead of author/year.

\usepackage{amsmath}

\newcommand{\cL}{{\cal L}}

\begin{document}

\special{papersize=8.5in,11in}
\setlength{\pdfpageheight}{\paperheight}
\setlength{\pdfpagewidth}{\paperwidth}

\conferenceinfo{CONF 'yy}{Month d--d, 20yy, City, ST, Country}
\copyrightyear{20yy}
\copyrightdata{978-1-nnnn-nnnn-n/yy/mm}
\copyrightdoi{nnnnnnn.nnnnnnn}

% Uncomment the publication rights you want to use.
%\publicationrights{transferred}
%\publicationrights{licensed}     % this is the default
%\publicationrights{author-pays}

\titlebanner{banner above paper title}        % These are ignored unless
\preprintfooter{short description of paper}   % 'preprint' option specified.

\title{Title Text}
\subtitle{Subtitle Text, if any}

\authorinfo{Name1}
           {Affiliation1}
           {Email1}
\authorinfo{Name2\and Name3}
           {Affiliation2/3}
           {Email2/3}

\maketitle

\begin{abstract}
Studies in the field of programming language design evaluation have shown that there exists a gap between small internal methods and large-scale surveys and evaluation methods. This leaves language designers and especially students developing new programming languages with no low-cost solution for language evaluation. In this report, as a starting point, we examine the applicability of the discount usability method on programming languages by surveying relevant literature. Our findings suggest that it is good for examining the IDE and compiler of a language, but is less suited for examining the language’s design. For this reason, we conducted a usability evaluation experiment on a language without using an IDE or a compiler. This lead to the creation of a new method which used the discount usability method and the IDA method as a basis. The usability evaluation experiment was carried out on Quorum using programmers with experience in C and C\#. Most of the problems found in the evaluation were related to the programmers pre-existing expectations of a language. When comparing our results with Quorum’s data, however, we found several discrepancies.  The results show that our evaluation method could serve as a low-cost way of evaluating programming languages for language designers. 
\end{abstract}

\category{D.3}{Programming Languages}{}
\category{H.5.2}{Information Interfaces and Presentation (e.g., HCI)}{User Interfaces}[User-centered design]

% general terms are not compulsory anymore,
% you may leave them out
\terms
Programming Language Design Evaluation, User Evaluation

\keywords
Quorum, Usability Evaluation, Language Design

\section{Introduction}

The text of the paper begins here.

\appendix
\section{Appendix Title}

This is the text of the appendix, if you need one.

\acks

Acknowledgments, if needed.

% We recommend abbrvnat bibliography style.

\bibliographystyle{abbrvnat}

% The bibliography should be embedded for final submission.

\begin{thebibliography}{}
\softraggedright

\bibitem[Smith et~al.(2009)Smith, Jones]{smith02}
P. Q. Smith, and X. Y. Jones. ...reference text...

\end{thebibliography}


\end{document}
