\section{Interview}

In the context of qualitative research, the interview is considered the most widely employed method. There are two main types of interviews associated with qualitative research - the \textit{structured} interview and the \textit{semi-structured} interview \cite{Bryman}. Qualitative interviewing is generally very different from  interviewing in quantitative research in the methods being employed. The qualitative approach is much less structured, focusing on formulation of research ideas rather than maximizing the validity of measurement of important concepts. Additionally, the discussion is tailored towards the interviewee's point of view more than than that of the researcher's and deviation in the responses is actually encouraged since it provides a degree of flexibility and thus, the possibility of rich and detailed answers. As already mentioned, the qualitative interview process can have two approaches. In the \textit{unstructured} approach the interviewer uses just a simple aide as prompts to where the conversation should lead. The interviewee is allowed to respond in a free manner and there could be a follow-up on interesting points. On the other hand, the \textit{semi-structured} approach warrants the interviewer to have a list of questions, addressing specific topics, working as an \textit{interview guide} but still leaving a certain degree of leeway in how the interviewee responds. Still, it is not mandatory to follow the guide as it is and questions do not have to have a specific order in being asked. 

Initially, for our usability evaluation experiment we made use of the first approach where rather than specific questions, we had an open discussion with the participants. The discussion started with addressing some general areas of interest (task completion, inheritance, information hiding etc.) and continued from there based on the responses from the participants and the direction they headed in. We had the intention to follow up with the same approach for the experiment involving our evaluation method as well, but based on the feedback from the pilot test, we decided to try a more structured approach. 
Therefore, as a part of the interview, we created a small questionnaire with several questions addressing the overall experience of the experiment.
These questions were not meant to replace the open discussion but rather serve as a baseline for the direction of the discussion and to ensure some specific areas were covered in the discussion.
Questions \#1, \#2 and \#3 were about the language and primarily served to get the participants own thoughts about it.
While there were some potential overlap in these questions, they could help some people talk more, and they helped categorise the feedback.
Question \#4 focused on getting feedback about our task and sample sheet.
Question \#5 asked about the experience of coding without a compiler since it is the biggest change for our method. Section \ref{subsection:interview questions} lists all the questions while the answers from the participants in the form of notes could be found in Appendix \ref{chapter:Interview notes}.

\subsection{Interview questions}
\label{subsection:interview questions}
\begin{enumerate}
\item What do you think about the language? Was it easy to learn?
\item Did you find some of the design odd?
\item How does Quorum relate to other languages you have experience in?
\item How did you find the tasks? Were they too challenging or too easy?
\item What do you think about coding without a compiler?
\end{enumerate}	