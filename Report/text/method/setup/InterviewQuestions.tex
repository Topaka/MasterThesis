\section{Interview}

In the context of qualitative research, the interview is considered the most widely employed method. There are two main types of interviews associated with qualitative research - the \textit{structured} interview and the \textit{semi-structured} interview \todo{ref here about social methods}. Qualitative interviewing is generally very different from  interviewing in quantitative research in the methods being employed. The qualitative approach is much less structured, focusing on formulation of research ideas rather than maximizing the validity of measurement of important concepts. Additionally, the discussion is tailored towards the interviewee's point of view more than than that of the researcher's and deviation in the responses is actually encouraged since it provides a degree of flexibility and thus, rich and detailed answers.

As a part of the interview, we created a small questionnaire with several questions addressing the overall experience of the experiment.
These questions were not meant to replace the open discussion but rather serve as a baseline for the direction of the discussion and to ensure some specific areas were covered in the discussion.
The first three questions were about the language and primarily served to get the participants own thoughts about it.
While there were some potential overlap in these questions, they could help some people talk more, and they helped categorise the feedback.
The second question focused on getting feedback about our task and sample sheet.
The third questions asked about the experience of coding without a compiler since it is the biggest change for our method.

\subsection{Interview questions}
\begin{enumerate}
\item What do you think about the language? Was it easy to learn?
\item Did you find some of the design odd?
\item How does Quorum relate to other languages you have experience in?
\item How did you find the tasks? Were they too challenging or too easy?
\item What do you think about coding without a compiler?
\end{enumerate}	