\chapter{Experiment Setup}
Quorum is an evidence-based language that uses results from experiments with novice programmers to decide on language design decisions.
Since our participants were computer science and software students who are all experienced programmers, we expected there would be some errors from this targeting mismatch for us to analyse.
As we are still focusing on the problems encountered during the process rather than the resulting code, we decided to record the experiments.
To make recording easier we decided to use a text-editor on a computer.
The text-editor we used was notepad++\cite{Notepad}.
Notepad++ has some features to assist programming, most notably an auto-completer which uses words already written in the text as suggestions.
However since these features are language-agnostic and the auto-completer would only prevent false positives from minor typos and not language misunderstandings, this was deemed acceptable.%or a good thing or somethinglike that
Since we wanted to allow for a more flexible schedule for participants, to be able to attract more of them, we decided not to use the usability lab.
Instead the tests were conducted using the laptop of either of us in the canteen of Cassiopeia.
To record the screen Microsoft Game DVR was used.
Due to the poor quality of the inbuilt laptop microphones, a smartphone was used to record the audio.%maybe just the last part

The process had two parts:
For the first part the participant was given a task sheet and a sample sheet and asked write programs that could complete the tasks on the task sheet.
The participant was told not to worry too much about the code being written correctly since it would not be compiled anyway, but to still try to get as close as they could.
During the test the participant could freely ask the facilitator for assistance, though the facilitator preferred to answer by referencing the sample sheet if possible.
The facilitator would also occasionally act as a surrogate compiler by telling the subject about some identified errors.%maybe noting a preference for focusing on non language specific errors though i doubt we followed that
The first part would take about an hour.
For the second part the subject would be interviewed about the language and the test.
There was an interview sheet with some questions to be answered during the interview, but other than that they were more informal talks.
During this part the subject would usually be made aware of most of the otherwise unmentioned errors, to be able to provide a more informed discussion.
\section{Participants sample}
\label{section:Participant sample}
All the participants taking part in the usability evaluation and the evaluation method were experienced programmers. They all had very similar age and occupation - Computer science students from the 4th semester and up, with programming experience in C and C\# as well as other programming languages ( e.g. Java, F\#, Python, Pascal).

We had six participants taking part in the main experiment, who are numbered participant \#3 to participant \#8.
Participant \#1 was the pilot test participant, we used to improve our setup before the first test.
Participant \#2 has been omitted from all the direct evaluation results as he was not an experienced programmer and therefore does not fit the target group of the experiment.
We have kept his data however as it shows some interesting pointers for problems encountered by novice programmers.

\chapter{Experiment Experiences}
In this chapter the observations and experiences made during the process of conducting the experiments will be described.

\section{Task Experiences}
Apart from the pilot test, we found the tasks to strike a good balance between difficulty and relative time it takes to solve them. Although we did not consider time as a sensitive factor for the experiment, we still wanted to stay in a certain time frame amounting to around one hour. All of the participants, except one, managed to finish the tasks around this time frame. Additionally, they found the tasks challenging enough and good at conveying their intended purpose - using specific constructs from the target programming language. 

\section{Sample Sheet Experiences}
When first reviewing the sample sheet, most participants on skimmed the code samples instead of reading it thoroughly. The sample sheet would then be used as reference, which the participant would look in when they were in doubt about something. The participant could easily spend a considerable amount of time, shuffling through the three sample sheets looking for the code example that demonstrated what they were looking for. This was often alleviated by the participant simply asking the facilitator their questions instead, at which point the facilitator could point out where the code would be in the sample sheet. Occasionally the specific question would not be in the sample sheet, in which case the facilitator would usually answer dependant on their knowledge of Quorum. Usually these questions were about features not present in Quorum.

\section{Interview Questions Experiences}
Initially, the interview questions did not even have a written form. This changed after a few iterations since we thought that writing the questions down rather than just verbal pointers will give us a more structured approach and it was easier to keep track off during the interview process. For this reason, rather than focusing on specific points, we tried to keep their number to a minimum and encapsulate a given interview direction with each, which the participants can share their thoughts on freely. Although they had a certain degree of freedom in their answers, common similarities and points were still addressed. 