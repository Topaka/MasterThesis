\section{Tasks}
The idea of creating a tasks sheet was to create tasks that would explore various features of the language being tested. Although each task had an intended purpose with a clear goal, their design allows more than one possible solution which gave the participants the freedom to experiment with the language.
\begin{itemize}
\item The first task had the intended purpose of testing arithmetic expressions and the use of data types.
\item The second task had the purpose of testing containers in the language (such as arrays) and control structures.
It also tested responsible code modification since there was a certain degree of intended repetitiveness in the subtasks which warranted careful reusing of code segments .%might not be good to point out
\item The third task was for testing the concept of classes and inheritance. The design of this task was inspired by the task used in the C\# section of our usability experiment in section \ref{subsection:C Task}.
\item The final task was testing operations on strings, including the exercise of in-build actions specifically useful for splitting text segments.
\end{itemize}

\subsection{Task Sheet}
\textbf{Task 1}:\\
Imagine a simple supermarket billing system which can specify orders and calculate the total price of ordered items. For the sake of simplicity, we work with oranges and bananas as our products. Oranges cost 5\$ per piece and bananas 4\$ per piece, respectively. Create a system that:
\begin{itemize}
\item Can calculate the total price given a number of oranges and bananas bought.
\item Adds a different price for buying a specific amount of an item 
\item Make triplets of oranges cost 10\$ in total instead of 15\$
\item Make 5 bananas cost 10\$ instead of 20\$
\item Adds a discount of 10\% to the total price for regular customers
\end{itemize}

\textbf{Task 2}:\\
Imagine you have 2 football teams and each team has an equal amount of players. Each player has both his first and last name written down as well as their age. Try to find the following things:
\begin{itemize}
\item 2 or more players with the same first or last name in the same team
\item 2 or more players with the same first or last name across the two teams
\item 2 or more players with the same first name and age in the same team
\end{itemize}

\textbf{Task 3}:\\
Imagine you have a simple Role playing game. You have a base character which can be specialized in different classes such as Warrior, Mage etc. Every character has a certain amount of hitpoints and has the ability to attack other characters. 
\begin{itemize}
\item Create a system for characters who all have:
\begin{itemize}
\item Hit points and the ability to replenish them
\item The ability to attack other characters
\end{itemize}
\item Allow a character to have a specific class
\item Add a specific unique resource to every class (Warriors get fury, Mages get mana)
\item Add a special unique attack to every class (Warriors get “Execute”, Mages get “Fireball” etc.)
\begin{itemize}
\item These unique attacks spend the unique resource, respectively (e.g. Fireball costs 10 mana) 
\end{itemize}
\item Add the ability for every class to replenish their unique resource.
\end{itemize}

\textbf{Task 4}:\\
For some given text (for example your full name), write a procedure which:
\begin{itemize}
\item Prints the text in reverse order
\item Prints the letters from the text in an alphabetical order
\item Finds if there are duplicate letters in the text and if there are, list how many are duplicated (e.g. “Tommy” will give the result of 1, while “Christensen” has 3) 
\end{itemize}