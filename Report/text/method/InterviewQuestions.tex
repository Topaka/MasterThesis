\section{Interview}
For the interview part of the test, we created some questions.
These questions were not meant to replace the open discussion but rather serve as a baseline for the discussion and to ensure some areas were covered in the discussion.
The first three questions were about the language and primarily served to get the participants own thoughts about it.
While there were some potential overlap in these questions, they could help some people talk more, and they helped categorise the feedback.
The second question focused on getting feedback about our task and sample sheet.
The third questions asked about the experience of coding without a compiler since it is the biggest change for our method.

\subsection{Observations}
\todo{move this to results somewhere}
\begin{itemize}
\item \textbf{The use of Quorum as a programming language} - the majority of the participants (\#3, \#4, \#5, \#8) found Quorum an easy to use and understand language. Additionally, some compared it and found it similar to other languages such as C, C\#, Pascal and Python and generally less verbose than standard OO languages they had experience in(e.g. C\#,Java). 
\item \textbf{Managing scoping rules by the use of the end keyword instead of brackets} - 
\item \textbf{Quorum uses the colon (":") notation instead of dot(".")}
\item \textbf{The lack of common control statements such as for and for-each loops was confusing} - 
\end{itemize}

\subsection{Interview questions}
\begin{enumerate}
\item What do you think about the language? Was it easy to learn?
\item Did you find some of the design odd?
\item How does Quorum relate to other languages you have experience in?
\item How did you find the tasks? Were they too challenging or too easy?
\item What do you think about coding without a compiler?
\end{enumerate}	