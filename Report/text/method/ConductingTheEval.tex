\chapter{Conducting the evaluation}
To test our method we used it on a language that was unlikely to be already used by any of our participants.
The language we used was the language Quorum\todo{cite quorumlang.org or something like that}.
Quorum is an evidence-based language that uses results from experiments with novice programmers to decide their language design decisions.
Since our participants were computer science and software students who are all experienced programmers, we expected there would be some errors from this targeting mismatch for us to analyse.
As we are still focusing on the problems encountered during the process rather than the resulting code, we decided to record the experiments.
To make recording easier we decided to use a text-editor on a computer.
The text-editor we used was notepad++\todo{maybe site their main page}.
Notepad++ has some features to assist programming, most notably an auto-completer which uses words already written in the text as suggestions.
However since these features are language-agnostic and the auto-completer would only prevent false positives from minor typos and not language misunderstandings, this was deemed acceptable.%or a good thing or somethinglike that
Since we wanted to allow for a more flexible schedule for participants, to be able to attract more of them, we decided not to use the usability lab.
Instead the tests were conducted using the laptop of either of us in the canteen of Cassiopeia.
To record the screen Microsoft Game DVR was used.
Due to the poor quality of the inbuilt laptop microphones, a smartphone was used to record the audio.%maybe just the last part

The process had two parts:
For the first part the participant was given a task sheet and a sample sheet and asked write programs that could complete the tasks on the task sheet.
The participant was told not to worry too much about the code being written correctly since it would not be compiled anyway, but to still try to get as close as they could.
During the test the participant could freely ask the facilitator for assistance, though the facilitator preferred to answer by referencing the sample sheet if possible.
The facilitator would also occasionally act as a surrogate compiler by telling the subject about some identified errors.%maybe noting a preference for focusing on non language specific errors though i doubt we followed that
The first part would take about an hour.
For the second part the subject would be interviewed about the language and the test.
There was an interview sheet with some questions to be answered during the interview, but other than that they were more informal talks.
During this part the subject would usually be made aware of most of the otherwise unmentioned errors, to be able to provide a more informed discussion.