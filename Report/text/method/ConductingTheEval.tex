\chapter{Setup}
To test our method we used it on a language that was unlikely to be already used by any of our participants.
The language we used was the language Quorum\todo{cite quorumlang.org or something like that}.
Quorum is an evidence-based language that uses results from experiments with novice programmers to decide their language design decisions.
Since our participants were computer science and software students who are all experienced programmers, we expected there would be some errors from this targeting mismatch for us to analyse.
As we are still focusing on the problems encountered during the process rather than the resulting code, we decided to record the experiments.
To make recording easier we decided to use a text-editor on a computer.
The text-editor we used was notepad++\todo{maybe site their main page}.
Notepad++ has some features to assist programming, most notably an auto-completer which uses words already written in the text as suggestions.
However since these features are language-agnostic and the auto-completer would only prevent false positives from minor typos and not language misunderstandings, this was deemed acceptable.%or a good thing or somethinglike that
Since we wanted to allow for a more flexible schedule for participants, to be able to attract more of them, we decided not to use the usability lab.
Instead the tests were conducted using the laptop of either of us in the canteen of Cassiopeia.
To record the screen Microsoft Game DVR was used.
Due to the poor quality of the inbuilt laptop microphones, a smartphone was used to record the audio.%maybe just the last part

The process had two parts:
For the first part the participant was given a task sheet and a sample sheet and asked write programs that could complete the tasks on the task sheet.
The participant was told not to worry too much about the code being written correctly since it would not be compiled anyway, but to still try to get as close as they could.
During the test the participant could freely ask the facilitator for assistance, though the facilitator preferred to answer by referencing the sample sheet if possible.
The facilitator would also occasionally act as a surrogate compiler by telling the subject about some identified errors.%maybe noting a preference for focusing on non language specific errors though i doubt we followed that
The first part would take about an hour.
For the second part the subject would be interviewed about the language and the test.
There was an interview sheet with some questions to be answered during the interview, but other than that they were more informal talks.
During this part the subject would usually be made aware of most of the otherwise unmentioned errors, to be able to provide a more informed discussion.

\section{Tasks}
The idea with the tasks was to create tasks that would explore various features of the language.
The first task was for testing arithmetic expressions.
The second task was for testing arrays and control structures.
It also tested responsible code modification.%might not be good to point out
The third task was for testing classes and inheritance.
The final task was testing operation on strings, though it had some overlap with task 2 and was mostly used to fill time if necessary.
\todo{might want to put these in an itemize}
The task sheet given looked like this:%maybe this is useless

\subsection{Task Sheet}
\textbf{Task 1}:\\
Imagine a simple supermarket billing system which can specify orders and calculate the total price of ordered items. For the sake of simplicity, we work with oranges and bananas as our products. Oranges cost 5\$ per piece and bananas 4\$ per piece, respectively. Create a system that:
\begin{itemize}
\item Can calculate the total price given a number of oranges and bananas bought.
\item Adds a different price for buying a specific amount of an item 
\item Make triplets of oranges cost 10\$ in total instead of 15\$
\item Make 5 bananas cost 10\$ instead of 20\$
\item Adds a discount of 10\% to the total price for regular customers
\end{itemize}

\textbf{Task 2}:\\
Imagine you have 2 football teams and each team has an equal amount of players. Each player has both his first and last name written down as well as their age. Try to find the following things:
\begin{itemize}
\item 2 or more players with the same first or last name in the same team
\item 2 or more players with the same first or last name across the two teams
\item 2 or more players with the same first name and age in the same team
\end{itemize}

\textbf{Task 3}:\\
Imagine you have a simple Role playing game. You have a base character which can be specialized in different classes such as Warrior, Mage etc. Every character has a certain amount of hitpoints and has the ability to attack other characters. 
\begin{itemize}
\item Create a system for characters who all have:
\begin{itemize}
\item Hit points and the ability to replenish them
\item The ability to attack other characters
\end{itemize}
\item Allow a character to have a specific class
\item Add a specific unique resource to every class (Warriors get fury, Mages get mana)
\item Add a special unique attack to every class (Warriors get “Execute”, Mages get “Fireball” etc.)
\begin{itemize}
\item These unique attacks spend the unique resource, respectively (e.g. Fireball costs 10 mana) 
\end{itemize}
\item Add the ability for every class to replenish their unique resource.
\end{itemize}

\textbf{Task 4}:\\
For some given text (for example your full name), write a procedure which:
\begin{itemize}
\item Prints the text in reverse order
\item Prints the letters from the text in an alphabetical order
\item Finds if there are duplicate letters in the text and if there are, list how many are duplicated (e.g. “Tommy” will give the result of 1, while “Christensen” has 3) 
\end{itemize}
\section{Samples}
The idea behind the sample sheet was to provide examples of code that the participant could use to learn what was necessary from the language in order to solve the tasks.
Working samples of code were used since it tends to give a more wholesome picture of how the code should look, without resorting to detailed description of how everything works.
The samples often included smaller details, not specifically relevant for what was explained at the time in order to cut down on the number of samples needed.
An example of this is the \lstinline!Paws()! method in the classes sample being used to demonstrate returning values from methods.
The constructors in Quorum were omitted from the samples as they are optional and their functionality does not support using parameters which is what most of our participants would consider the purpose of constructors.
The sample sheet can be found in \appref{chapter:sampleSheet}.

\section{Interview Questions}
\begin{enumerate}
\item What do you think about the language? Was it easy to learn?
\item Did you find some of the design odd?
\item How does Quorum relate to other languages you have experience in?
\item How did you find the tasks? Were they too challenging or too easy?
\item What do you think about coding without a compiler?
\end{enumerate}	