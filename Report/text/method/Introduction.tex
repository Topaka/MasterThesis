\chapter{Introduction}
\label{chapter:MethodIntro}
For our method we want to focus on evaluating the language design of a programming language.
Following our previous observations on the IDE's impact on the result, the next step would be to remove the IDE from the test.
Since the idea is to focus on how well the language fits with how one want to code regardless of any tools, we also want to avoid using a compiler or other specialized tools for the language.%maybe it is not clear how the compiler helps
What this means for the method is that the test can be done in any text-editor or indeed on physical paper.
This has the added bonus of the it being possible to do the test before a compiler has been created for the language.

Another thing our previous test showed was the high difficulty of suddenly programming in a language where the rules are not well known to the user.
Since a large portion of the intended use of this method is on new programming languages this is also an important concern for the method.
Our idea for addressing this is to, along with the task sheet, add a sample sheet, which would contain examples of code written in the language with an explanation of the codes functionality.

To test our method we used it on a language that was unlikely to be already used by any of our participants.
The language we used was the language Quorum\cite{Quorum}.