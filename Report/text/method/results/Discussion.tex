\chapter{Discussion}
The results from Chapter \ref{chapter:Results} address some potential problems when working with a language such as Quorum. This section will elaborate on how that can be extended to other programming languages and how our method could be used in a customized manner.

Comparing our results with Quorum's evidence has shown that our method gets comparable results to other methods, but with a significantly lower amount of participants.
Most of Quorum's evidence about experienced programmers has, however, been focused on just the syntax.
This means that most of the comparable data lies in our cosmetic errors, which are usually the least interesting problems from a usability standpoint.
The more serious problems tend to either contradict or not be addressed by Quorum's evidence, though in most cases, this is a result of the mismatch in target group between novice and experienced programmers.
One noticeable problem we encountered in the execution of our test was participants freezing at the very beginning.
They were unsure how to start as they could not figure out what format of the solution they should use.
For most of the participants this was not a big hurdle, as they would either just pick one way of doing it or consult the facilitator.
However, for some participants, having a discussion while programming was unnatural.
One way to prevent this could be to have some pre-written code that the participant should instead fill out.
It does however sacrifice some of the potential data about the language that a more free-form task can give.

After conducting both of our experiments, a very interesting observation can be made that an Integrated Development Environment (IDE), when used in conjunction with a programming language, contributes primarily to resolving cosmetic problems and mistakes associated strictly with the syntax of the given language. However, An IDE does not contribute that much to the facilitation process if the user gets "stuck", a state attributed to the critical problems from the IDA evaluation. Additionally, we noticed that even if people are experienced with a specific paradigm (participant \#2 from the usability evaluation in regards to C\#), they can still get into a position where they cannot continue with the experiment, given that they have to solve a task in a paradigm, different from what they are familiar with (participant \#2' experience with the F\# task). Further observations from the evaluation method showed that when people familiar with a specific paradigm (imperative, object-oriented) have to make use of an unfamiliar language, supporting such paradigm (Quorum is both imperative and object-oriented), they tend to disregard the syntax of the new language in place of a language they are familiar with. In the case with Quorum, most of the participants (\#1, \#3, \#4, \#5, \#6, \#7 and \#8) made use of syntax native to languages such as Java and C\#, supporting the same paradigms as Quorum.

\section{Threats to validity}
Conducting the experiment had some informal and qualitative conditions, which makes the validity face some threats. This section will describe the most prevalent of such threats.
\begin{itemize}
\item Participant sample - Although we collected some qualitative results, the experiment did not have a good representation of the general populace. The participant sample involved a small group, with very similar educational backgrounds, occupation, age and geological location as mentioned in section \ref{section:Participant sample}. The involvement of a bigger and more diverse group might skew the results in a different direction.
\item Facilitating the participants - Since we did neither use a compiler nor an IDE for the experiment, the facilitator had to help on several occasions and the participants refered to him rather than the sample sheet.
\end{itemize}


\section{Tasks \& Samples}
It is difficult to create good tasks, as we have experienced in our experiments.
Firstly we had to determine which parts of the language we wanted to test.
For our C\# test it was object-oriented programming, for our F\# test it was recursive functions and for our quorum test it was arithmetic operations, operations on containers, objects and classes and operations on strings.
Then we tried to build a scenario that would give us tasks that exercise that decision.
The specific scenario was not too important but having one helps convey the task, and gives some ideas for examples and names, which eases some unintended create burden on the participant.
Next we had to decide what how to formulate the tasks.
Here we ran into the problem of how strictly one wants the process to go.
For our first tests on C\# and F\# we intentionally kept the tasks vague to ensure we would not corrupt the data.
This however allowed the massive variance in task completion we had between our two participants.
It is possible this could have been avoided if we had more strictly defined tasks, however that would also have caused us to lose the data we got about the intuitiveness of using inheritance.
For our test on Quorum, we used more smaller tasks in several different scenarios instead of one big scenario as in the previous tests.
This meant we had to have a more strict formulation of the tasks.
Otherwise there was a higher risk of any tasks taking up all the time as it was interpreted to be larger or more complex than intended, which we discovered in our pilot test.
However even with a stricter formulation we ran into participants overcomplicating tasks by interpreting them in a larger scope of the scenario than we had intended.
At the same time we also encountered some participants not being able to start on the first task as they simply did not know where to start on a blank piece of paper.
One way to make the formulation even stricter without dictating what to write could be to provide the skeleton of the solution we expect.
This might also fix the issue with the blank page paralysis as it gives the participant a starting point to work from.
It does of course lose the data about how the participant would structure the solution as we would have done so for them.


remember to use the output instead of return error