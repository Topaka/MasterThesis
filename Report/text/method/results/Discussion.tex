\chapter{Discussion}
The results from Chapter \ref{chapter:Results} address some potential problems when working with a language such as Quorum. This section will elaborate on how that can be extended to other programming languages and how our method could be used in a customized manner.
\\

Comparing our results with Quorum's evidence has shown that our method gets comparable results to other methods but with a significantly lower amount of participants.
Most of Quorum's evidence about experienced programmers has however been focused on just the syntax.
This means that most of the comparable data lies in our cosmetic errors, which are usually the least interesting problems from a usability standpoint.
The more serious problems tend to either contradict or be not addressed by Quorum's evidence, though in most cases this is a result of the mismatch in target group between novice and experienced programmers.
One noticeable problem we encountered in the execution of our test was participants freezing at the very beginning.
They were unsure how to start as they were could not figure out what format of the solution they should use.
For most of the participants this was not a big hurdle as they would either just pick one way of doing it or consult the facilitator.
However for some participants, having a discussion while programming was unnatural.
One way to prevent this could be to have some pre-written code that the participant should instead fill out.
this also would provide a more structured presentation of the task, which can help create a more uniform scope in the solution between participants\todo{have we highlighted the troublesome task scoping yet?}.
It does however sacrifice some of the potential data about the language that a more free-form task can give.

After conducting both of our experiments, a very interesting observation can be made that an Integrated Development Environment (IDE), when used in conjunction with a programming language, contributes primarily to resolving cosmetic problems and mistakes associated strictly with the syntax of the given language. An IDE however, does not contribute that much to the facilitation process if the user gets "stuck", a state attributed to the critical problems from the IDA evaluation. Additionally, we noticed that even if people are experienced with a specific paradigm (participant \#2 from the usability evaluation in regards to C\#), they can still get into a position where they cannot continue with the experiment, given that they have to solve a task in a paradigm, different from what they are familiar with (participant \#2' experience with the F\# task). Further observations from the evaluation method showed that when people familiar a specific paradigm (imperative,object-oriented) have to make use of an unfamiliar language, supporting such paradigm (Quorum is both imperative and object-oriented), they tend to disregard the syntax of the new language in place of a language they are familiar with. In the case with Quorum, some of the participants (\#3,\#5,\#6 and \#7) made use of syntax native to languages such as Java and C\#, supporting the same paradigms as Quorum.
\section{Threats of validity}