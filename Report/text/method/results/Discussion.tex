\chapter{Discussion}
The results from Chapter \ref{chapter:Results} address some potential problems when working with a language such as Quorum. This section will elaborate on how that can be extended to other programming languages and how our method could be used in a customized manner.
\\
A lot of our results were unsurprising.
Quorum is a language that uses evidence about programming to design a language that is intuitive for novices.
Since our participants were experienced programmers though, it would be expected that a lot of the errors encountered would be related to this mismatch.
Especially the critical error with lacking constructors with parameters, showed a large mismatch in what an experienced programmer expected from a class compared to what was proven to be more user-friendly for a novice\todo{ref usability study of constructors}.
\todo{maybe start here instead as the above might be better for the comparison in results}
Comparing our results with Quorum's evidence has shown that our method gets comparable results to other methods but with a significantly lower amount of participants.
Most of Quorum's evidence about experienced programmers has however been focused on just the syntax.
This means that most of the comparable data lies in our cosmetic errors, which are usually the least interesting problems from a usability standpoint.
The more serious problems tend to either contradict or be not addressed by Quorum's evidence, though in most cases this is a result of the mismatch in target group between novice and experienced programmers.
One noticeable problem we encountered in the execution of our test was participants freezing at the very beginning.
They were unsure how to start as they were could not figure out what format of the solution they should use.
For most of the participants this was not a big hurdle as they would either just pick one way of doing it or consult the facilitator.
However for some participants, having a discussion while programming was unnatural.
One way to prevent this could be to have some pre-written code that the participant should instead fill out.
this also would provide a more structured presentation of the task, which can help create a more uniform scope in the solution between participants\todo{have we highlighted the troublesome task scoping yet?}.
It does however sacrifice some of the potential data about the language that a more free-form task can give.

\section{Threats of validity}