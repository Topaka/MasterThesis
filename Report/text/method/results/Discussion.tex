\chapter{Discussion}
The results from Chapter \ref{chapter:Results} address some potential problems when working with a language such as Quorum. This section will elaborate on how that can be extended to other programming languages and how our method could be used in a customized manner.

\section{Quorum analysis}
In their empirical experiments, Stefik and Gellenbeck \cite{EmpStudiesonStimuli} gathered many statistically significant results regarding keyword choice. Given that they try to primarily address visually impaired people and novice programmers, selecting the most intuitive words seems like a logical choice. For one of their  experiments \cite{EmpStudiesonStimuli}, they divided the participants in two groups - novices and experienced programmers to find out if there is a significant discrepancy in the results between the two groups. Every keyword choice was ranked in two tables by mean value and standard deviation. Since we conducted our method on people with programming experience, we are primarily interested in the results of the second group. For the purpose of our evaluation method, we will not mention every single word choice they rated but rather the ones which are coinciding with the problems we identified from the IDA evaluation in section \ref{section: Problem Categorization}. Additionally, we would also relate two of their empirical studies from Stefik and Siebert \cite{Empiricalinvestigation} and their findings about keyword choices. This would help us to make a comparison between the authors findings and the results from our evaluation, essentially giving an additional degree of credibility to the method if similarities are found.

\begin{itemize}
\item In the results for the concepts of and,or and xor, Stefik and Gellenbeck \cite{EmpStudiesonStimuli} found out that for the and concept,  using a single ampersand with and performed quite well, statistically comparable to and. Contrary to how many programming languages make use of - "\&" for bitwise and and "\&\&" for logical and, their results showed that these words are actually not that popular and thus not so intuitive to use. As for the logical or concept, the or keyword was placed first, being significantly better the the second highest one - the "||" operator, which is present in many popular programming languages. Last but not least, the xor logical operator, the or was rated highest which can be attributed to the fact that the participants did not know how to call an operation which ?took a behavior when one condition was true but not both?.
\item Stefik and Gellenbeck \cite{EmpStudiesonStimuli} settled on using a single equals ("=") sign for assignment statements and for testing equality since that is what they thought would make most sense. Although this might be true for the novice group (single equals ("=") sign was ranked highest), the experienced programmers group did not even rate the single equals in the top 3, rating the double equals ("==") as highest instead. This was not verified until one of the later empirical studies by Stefik and Siebert \cite{Empiricalinvestigation}.
\item For the concept of "Taking a behaviour" the authors considered several word choices such as function,action and method. The novices ranked the action word the highest, while the experienced programmers - operation, followed by action,method and function. However, the authors admit that this particular results should be further investigated since the participants might have understood the description of the concept as something other than completely capturing the idea of a function.
\item Quorum makes use of the keyword repeat over for or while, or cycle (Sanchez and Flores (ref here)) following a study which shows that repeat represents the concept of iteration significantly better than the the aforementioned words. (ref here)
\end{itemize}

Based on the results from Stefik and Siebert \cite{Empiricalinvestigation}, the programmers group found "==" to be intuitive as the boolean equals operator, which matches our observation of the participants often using "==" instead of "=" notation. For many of our problems where our participants used the wrong syntax, \cite{Empiricalinvestigation} did have comparable results were both the wrong and the correct syntax was found intuitive by their programmers group. These are: 
\begin{itemize}
\item Dot (".") versus colon (":")
\item using an aggregate operator (x += 1) versus an arithmetic operator (x = x + 1)
\item logical and (\&\&) versus and
\item logical or (||) versus or 
\item data types wording (float versus number, string versus text and bool versus boolean). 
\end{itemize}

It is possible that a lot of these cases were the result of a participant just glancing over the sample sheet. Since the constructs looked intuitive, they did not notice or remember that it was different and thus just used the syntax they were used to. Interestingly, our results about the looping constructs contradicts the results from the paper. Our participants often lamented the lack of a for or foreach loop and had a lot of errors in using the iterator for the while loop. This is in contrast to the papers results where their programmers did not find "for" among the most intuitive and found foreach among the least intuitive keywords for looping. However, the results are not directly contradictory as the paper?s questions about intuitiveness was focused on the syntax, while our participants problems were more about lacking the functionality of a looping construct with inbuilt iterator handling. A more direct contradiction is that our participants often found the repeat keyword unnecessary, despite the paper listing it as one of the most intuitive keywords for looping. This could be a side effect of us only demonstrating the repeat while loop in our sample sheet, since that loop looks exactly like the while loop they are used to but with an extra keyword in front. 


 





\section{Threats of validity}