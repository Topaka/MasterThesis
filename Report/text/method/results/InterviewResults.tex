\section{Interview Results}
During the interview we collected feedback addressing various key points such as the usability of programming languages (in particular Quorum), programming without a compiler or the help of an IDE, impressions of how effective some constructs are and how they can be improved. The most common observations among the participants were:

\begin{description}
\item[The use of Quorum as a programming language] - the majority of the participants (\#3, \#4, \#5, \#8) found Quorum an easy to use and understand language. Additionally, some compared it and found it similar to other languages such as C, C\#, Pascal and Python and generally less verbose than standard OO languages they had experience with (e.g. C\#,Java). 
\item[Managing scoping rules by the use of the "end" keyword instead of brackets] - Generally, most of the participants found the use of the end keyword for defining scopes very confusing. Participants \#4 and \#6 preferred the use of brackets, similar to OO languages they were familiar with (e.g. Java and C\#), while participants \#7 and \#8 preferred indentation similar to languages like Python. Participant \#7 further suggested to extend the end construct to begin-end, similar to Pascal which would make the language more user-friendly for novices. 
\item[Quorum uses the colon (":") notation instead of dot(".")] - Given their prior experience with programming languages where the dot (".") notation is common, participants \#4, \#6 and \#7 found it confusing to use the  colon (":") notation instead. This confusion was further reinforced by the fact that the colon could be used in different scenarios yet the dot notation is still used when calling libraries.
\item[The lack of common control statements such as for and for-each loops was confusing] - The lack of common control statements such as \textit{for} or \textit{for-each} loops in Quorum seemed like a hurdle for the participants and consequently they found it not so intuitive to use the \textit{repeat while} construct as a substitute of that. This is evident by the fact that some of them completely forgot to include it in the loop's signature (participant \#3) or found it unnecessary altogether (participant \#7).
\end{description}