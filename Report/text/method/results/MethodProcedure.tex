\chapter{Method procedure}
\label{section:MethodSummary}
Following these experiences we have created the full method.
In summary, the steps of applying the method to a language are as following:

\begin{enumerate}
\item \textbf{Create tasks} These tasks are specific to the language, and should explore key features of the language. A useful tool to design tasks can be to create some scenarios you would expect a user to use your language in and what that user would need to do solve their task.
\item \textbf{Create a sample sheet} Based on the tasks, you now have a better idea of what a participant would need to know to solve those tasks. Keeping the sample sheet short or having a clear indexing of the samples can help participants browse the sample sheet. Having working code samples can help give a better understanding of the overall structure of code in the language.
\item \textbf{Estimate the task length} Taking time of how fast you can solve the tasks will give an idea of how long the experiment will take per participant. Do note that the participants will likely take longer to solve the tasks since they have to get acquainted with the language first. It can be okay to have more tasks than what you expect a participant to be able to solve, but the later tasks would need to explore less important features, and the participant needs to be made aware of not beeing expected to solve all of them.
\item \textbf{Prepare setup} The specifics of the setup can vary from a full blown usability lab to pen and paper. The advantage of a flexible setup, like pen and paper or a laptop with a text-editor, is that the convenience can make it more attractive for potential participants. Often the experiment will be recorded to better review the process of solving the tasks, in which case the necessary utilities for this needs to be prepared.
\item[(optional)] \textbf{Conduct a pilot test} A pilot test can let you discover and fix any problems in your tasks, sample sheet task estimate and setup before conducting the experiment on the full number of participants. It does however require an additional participant and some more time.
\item \textbf{Gather participants} The golden rule for number of participants is five. More than that and you tend to mostly only observe problems you already have observed though the confirmation is nice. Less than that and you tend to have a several of problems left undiscovered though some data is generally still better than none.
\item \textbf{Start the experiment} Make sure to tell the participant that it is the language being tested and not them, to alleviate some unnecessary nervousness.
\item \textbf{Keep the participant talking} Try to make the participant talk about what they are thinking about solving the task at hand. During this time the facilitator may answer any questions the participant have about the language. The facilitator should try to avoid talking about how to solve the tasks, but it may be necessary if the participant need help getting started (or stopped in cases of overcomplicating tasks). The facilitator will confirm when a task is done as the system won't give that kind of feedback.
\item \textbf{Interview the participant} After the test, have a brief interview with the participant where you can discuss the language, tasks etc. It can be useful to have some questions pre-written or create a questionnaire if there are many participants.
\item \textbf{Analyse data} After all the tests have been conducted, use the data to identify a list of problems encountered during the test. You can then categorize the problems using the following guidelines:
\begin{description}
\item[Cosmetic problems] are typos and small keyword and character differences that can easily be fixed by replacing the wrong part.
\item[Serious problems] are structural errors that usually impacts how the code is structured, but is usually small enough that it can be fixed with a few changes.
\item[Critical problems] are fundamental misunderstandings of how the language structures code and large structural errors that would require a revision of the algorithm.
\end{description}
Following this categorization you will now a prioritized list of things to improve on the language.
\end{enumerate}

\section{Interview Suggestions}
During the interview we collected several suggestions as to what could be done to improve the experiment.
These suggestions have been described here.

\begin{description}
\item[Providing a skeleton for the task solutions] One of the suggestions was, to add some code on the code sheet that would show the skeleton of the expected solution. The participant would then only have to worry about filling in the code for the functionality rather than how the solution should be structured.
\item[Adding a cheat sheet to the sample sheet] Another suggestions was to add an extra sheet to the sample sheet, that would contain just a list of all the keywords and constructs, to have a single page to look at when looking for a specific thing.
\item[Using separate pages for each task] A third suggestion was to have each task on a separate page, which would allow addition of some samples specific to that task on the page. In essence by having a smaller subset of the samples for each task, the amount of sample code to look through at any time would be reduced.
\end{description}