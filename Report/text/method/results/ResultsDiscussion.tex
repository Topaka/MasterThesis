\chapter{Results \& Discussion}

This section will highlight the results from the evaluation method and a discussion of how the data can serve as a future reference to language designers and people interested in conducting their own evaluation methods. Similarly to how the results were evaluated in Chapter \ref{chapter:ExperimenteEvaluation}, we make use of the IDA method \cite{IDA} to categorize the problems we identified based on their importance and severity.

The premise of the experiment is to try to divide the IDE from the language which does not warrant using a compiler. However, this might affect the categorization of problems and the impact they might have while completing the tasks. For this reason, we would try to reason where each problem should be and would it make a difference given that a compiler have been used instead.

\begin{table}[]
\centering
\renewcommand{\arraystretch}{1.5}
\caption{The table of identified problems categorised by severity\todo{missing problem about forgetting to import array library}}
\label{QuorumProblemResult}
\begin{tabular}{| p{5cm} | p{5cm} | p{5cm} |}
\hline
\textbf{Critical}                                          & \textbf{Serious}                                                   & \textbf{Cosmetic}                                                                                        \\ \hline
Not using the “end” keyword at all                         & Not using the “end” keyword to end the scope of if-statements      & Using colon (“:”) instead of dot (“.”)                                                                   \\ \hline
The lack of constructors in Quorum                         & Forgetting to increment the iterator in a repeat while loop        & The lack of lambda operators in Quorum (e.g.“+=”)                                                        \\ \hline
Misunderstanding the effect of Sort() on arrays of classes & The lack of common looping constructs (for-loops or foreach loops) & Using conditional AND and OR as “\&\&” and “||” instead of “and” and “or” keywords, as defined in Quorum \\ \hline
                                                           & Forgetting to import a library for containers (array)              & Using the “float” instead of “number” keyword                                                            \\ \hline
                                                           & Not using “elseif” to avoid having to close an additional scope    & Using “string” instead of the “text” keyword                                                             \\ \hline
                                                           & Not resetting inner loop iterator between loops                    & Writing “output” instead of “return” as the keyword for a return statement                               \\ \hline
                                                           &                                                                    & Using “==” in conditional statements instead of “=”                                                      \\ \hline
                                                           &                                                                    & Using “int” instead of “integer”                                                                         \\ \hline
                                                           &                                                                    & Using “bool” instead of “boolean”                                                                        \\ \hline
                                                           &                                                                    & Typos in library importing                                                                               \\ \hline
                                                           &                                                                    & Mistyping “integer” as “integar”                                                                         \\ \hline
                                                           &                                                                    & Accidentally used 0 instead of O in variable name                                                        \\ \hline
                                                           &                                                                    & Mistyped the “is” keyword as “ia”                                                                        \\ \hline
                                                           &                                                                    & Forgot to add the “repeat” keyword                                                                       \\ \hline
\end{tabular}
\end{table}

\textbf{Critical problems}
\begin{enumerate}
\item \textbf{Not using the “end” keyword at all} - this would affect the overall validity of the program because the scoping rules in Quorum are defined in conjunction with the “end” keyword. This shows a fundamental misunderstanding of how scoping works in the language
\item \textbf{The lack of constructors in Quorum} - Quorum does not support constructors which might problematic for the participants, having experience with other languages where this feature is mandatory. It both causes the participants to invoke syntax in the class that is not supported, and have difficulties instantiating classes. Since this is a significant difference in how the code should be structured, it is considered critical.
\item \textbf{Misunderstanding the effect of Sort() on arrays of objects} - The inbuilt Sort function for arrays does not have access to the properties of the objects and therefore does not sort them by any of those. This would have the consequence of code, written with the assumption that it works, be most likely wrong, which means recovery would require a full rewrite of the algorithm. This makes the problem critical. It is possible that with the use of a compiler the participant would discover and recover much easier, which could mean the problem would potentially be considered serious.
\end{enumerate}

\textbf{Serious problems}
\begin{enumerate}
\item \textbf{Not using the “end” keyword to end the scope of if-statements} - Although this problem looks similar to the first problem defined in Critical problems, the difference is that it is more likely to be an overlook than a misunderstanding of the scoping rules in Quorum. Also single-line if-statements might be present in other languages and not in Quorum.
\item \textbf{Forgetting to increment the iterator in a repeat while loop} - This could be considered an oversight on the participant’s part, attributed to how the repeat-while construct works in Quorum compared to how usually for-loops are used, and therefore it was not critical. However, it is still considered a serious problem because of the impact it has on the structural correctness of the code.
\item \textbf{The lack of common looping constructs (for-loops or foreach loops)} - This is considered a serious problem for few reasons. Firstly, it warrants the use of the repeat-while construct as a part of Quorum, which might not be so intuitive for people coming with backgrounds in other languages, where these constructs are present. Secondly, this might compound to the previous problem described in this section which would have a high impact on the validity of the written program. 
\item \textbf{Forgetting to import a library for containers (array)} - Containers in Quorum, and specifically arrays, have to imported first before being used. This is considered a serious problem since it might have a high impact on the validity of the program. 
\item \textbf{Not using “elseif” to avoid having to close an additional scope} - This problem is serious because it shows a lack of understanding the finer points of scoping in an if-else chain. Problematic since “else if” is also valid syntax, but carries unintended consequences.
\item \textbf{Not resetting inner loop iterator between loops} - Similarly to serious problem 2, this could be considered an oversight on the participant’s end due to previous experience with other programming languages.
\end{enumerate}

\textbf{Cosmetic problems}
\begin{enumerate}
\item \textbf{Using colon (“:”) instead of dot (“.”)} - This doesn’t affect how the language is structured and thought about but is only a exchanging a single character.
\item \textbf{The lack of lambda operators in Quorum (e.g.“+=”)} - This is considered a cosmetic problem since it does not affect the correctness of the program but it is rather a matter of convenience for the participants.
\item \textbf{Using conditional AND and OR as “\&\&” and “||” instead of “and” and “or” keywords, as defined in Quorum}  - This problem is considered cosmetic because the participants did not use the correct keywords in the context but had the proper intentions. This could be attributed to the simple matter of not properly reading the sample sheet to find the proper keywords and using the ones they know from other languages instead.
\item \textbf{Using the “float” instead of “number” keyword} - This is considered a cosmetic problem because it does not have a big impact on the program’s correctness but rather is using a naming convention from other programming languages
\item \textbf{Using “string” instead of the “text” keyword} - this is the same as with the previous cosmetic problem
\item \textbf{Writing “output” instead of “return” as the keyword for a return statement} - This is cosmetic because it is mostly a result of our sample sheet using “output” often, while the participants were more commonly expected to write code returning something and a familiarity with the “return” keyword from other languages.
\item \textbf{Using “==” in conditional statements instead of “=”} - Similarly to previous cosmetic problems, the main reason behind this problem is that most of the participants had experience with other programming languages where the “==” notation is common and in turn had a particular bias against using the “=” notation.
\item \textbf{Using “int” instead of “integer”}  - same as with cosmetic problem 4
\item \textbf{Using “bool” instead of “boolean”} - same as with cosmetic problem 4
\item \textbf{Typos in library importing} - This is a simple case of having small typos when writing the import code. Easily fixed and a cosmetic problem.
\item \textbf{Mistyping “integer” as “integar”}  - this is a cosmetic problem since it is a siimple typing mistake and it does not have any impact on the validity of the program.
\item \textbf{ Accidentally used 0 instead of O in variable name}  - Again another small typo and therefore cosmetic.
\item \textbf{Mistyped the “is” keyword as “ia”}  - see problem 12
\item \textbf{Forgot to add the “repeat” keyword} - This problem is considered cosmetic since it does not have a significant impact on the correctness of the program.

\end{enumerate}

\section{Interview Results}
During the interview we collected feedback addressing various key points such as the usability of programming languages (in particular Quorum), programming without a compiler or the help of an IDE, impressions of how effective some constructs are and how they can be improved. The most common observations among the participants were:

\begin{description}
\item[The use of Quorum as a programming language] - the majority of the participants (\#3, \#4, \#5, \#8) found Quorum an easy to use and understand language. Additionally, some compared it and found it similar to other languages such as C, C\#, Pascal and Python and generally less verbose than standard OO languages they had experience with (e.g. C\#,Java). 
\item[Managing scoping rules by the use of the \lstinline!end! keyword instead of brackets] - Generally, most of the participants found the use of the \lstinline!end! keyword for defining scopes very confusing. Participants \#4 and \#6 preferred the use of brackets, similar to OO languages they were familiar with (e.g. Java and C\#), while participants \#7 and \#8 preferred indentation similar to languages like Python. Participant \#7 further suggested to extend the \textit{end} construct to \textit{begin-end}, similar to Pascal which would make the language more user-friendly for novices. 
\item[Quorum uses the colon (\lstinline!:!) notation instead of dot(\lstinline!.!)] - Given their prior experience with programming languages where the dot (\lstinline!.!) notation is common, participants \#4, \#6 and \#7 found it confusing to use the  colon (\lstinline!:!) notation instead. This confusion was further reinforced by the fact that the dot notation is still used when calling libraries.
\item[The lack of common control statements such as for and for-each loops was confusing] - The lack of common control statements such as \textit{for} or \textit{for-each} loops in Quorum seemed like a hurdle for the participants and consequently they found it not so intuitive to use the \textit{repeat while} construct as a substitute of that. This is evident by the fact that some of them completely forgot to include it in the loop's signature (participant \#3) or found it unnecessary altogether (participant \#7).
\end{description}