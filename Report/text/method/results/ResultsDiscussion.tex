\chapter{Results}
\label{chapter:Results}
This Chapter will highlight the results from the evaluation method. Initially, we conducted a pilot test with one participant in order to test the overall setup of the experiment and get some pointers of what areas could be further improved. Similarly to how the results were evaluated in \chapref{chapter:ExperimentEvaluation}, we made use of the IDA method \cite{IDA} to categorize the problems we identified based on their importance and severity. Furthermore, we compared our results with the exiting findings about the evaluation of Quorum in order to prove the validity of our method.

\section{Pilot Test}
The pilot test was conducted on a single participant which served the purpose of giving us some feedback on how we can improve the test setup.
One of the bigger things was that the coding part of the test took two hours rather than the one hour we had intended despite skipping one of the tasks.
Following the pilot test we made several changes to our setup:
\begin{itemize}
\item Task 1 in the task sheet was largely rewritten to have more concrete examples and to try and focus on the simpler calculations.
This was due to our participant starting to spend a lot of time working on constructs for a larger shopping system instead of focusing on the arithmetic core of the task we had intended.
\item Task 2 was shortened from 5 subtasks to three subtasks due to the repetitive nature of the subtasks and to cut down on time spent.
\item The participant skipped task 3(now task 4) due to its similarity with task 2 which caused us to swap its position with task 4(now task 3).
This was also helped by the observation that getting data about classes in Quorum was deemed more important than operations on strings.
\item In the discussion the access modifiers for methods and properties in classes was brought up which caused us to realise this was not discussed in the sample sheet.
Some description and examples of access modifiers for classes was then added to the end of the sample sheet.
\item When conducting the interview we realised that we were unprepared as we did not know what we wanted to discuss.
This caused us to create the interview questions to provide a guideline for things to discuss.
\end{itemize}

\section{Problems categorisation}
\label{section: Problem Categorization}
The premise of the experiment is to try to divide the IDE from the language which does not warrant using a compiler.
However, since this means the system cannot give any feedback, The participant would not spend time on problems, which means the usual rules for categorisation, used in the IDA method, can not apply.
For this reason, we would try to reason where each problem should be, and would it make a difference if a compiler had been used instead.
The general guidelines we have used for this categorisation are:
\begin{description}
\item[Cosmetic problems] are typos and small keyword and character differences that can easily be fixed by replacing the wrong part.
\item[Serious problems] are structural errors that usually impacts how the code is structured, but is usually small enough that it can be fixed with a few changes.
\item[Critical problems] are fundamental misunderstandings of how the language structures code and large structural errors that would require a revision of the algorithm.
\end{description}

\begin{table}[!h]
\centering
\renewcommand{\arraystretch}{1.5}
\label{QuorumProblemResult}
\begin{tabular}{| p{5cm} | p{5cm} | p{5cm} |}
\hline
\textbf{Critical}                                          & \textbf{Serious}                                                   & \textbf{Cosmetic}                                                                                        \\ \hline
Not using the \lstinline!end! keyword at all               & Not using the \lstinline!end! keyword to end the scope of if-statements      & Using colon (\lstinline!:!) instead of dot (\lstinline!.!)                               \\ \hline
The lack of constructors with parameters in Quorum         & Forgetting to increment the iterator in a repeat while loop        & The lack of aggregate operators in Quorum (e.g.\lstinline!+=!)                                           \\ \hline
Misunderstanding the effect of \lstinline!Sort()! on arrays of classes & The lack of common looping constructs (for-loops or foreach loops) & Using conditional AND and OR as \lstinline!&&! and \lstinline!||! instead of \lstinline!and! and \lstinline!or! keywords, as defined in Quorum \\ \hline
                                                           & Forgetting to import a library for containers (array)              & Using the \lstinline!float! instead of \lstinline!number! keyword                                                            \\ \hline
                                                           & Not using \lstinline!elseif! to avoid having to close an additional scope    & Using \lstinline!string! instead of the \lstinline!text! keyword                                \\ \hline
                                                           & Not resetting inner loop iterator between loops                    & Writing \lstinline!output! instead of \lstinline!return! as the keyword for a return statement           \\ \hline
                                                           &                                                                    & Using \lstinline!==! in conditional statements instead of \lstinline!=!                                  \\ \hline
                                                           &                                                                    & Using \lstinline!int! instead of \lstinline!integer!                                                     \\ \hline
                                                           &                                                                    & Using \lstinline!bool! instead of \lstinline!boolean!                                                    \\ \hline
                                                           &                                                                    & Typos in library importing                                                                               \\ \hline
                                                           &                                                                    & Mistyping \lstinline!integer! as \lstinline!integar!                                                     \\ \hline
                                                           &                                                                    & Accidentally used \lstinline!0! instead of \lstinline!O! in variable name                                \\ \hline
                                                           &                                                                    & Mistyped the \lstinline!is! keyword as \lstinline!ia!                                                    \\ \hline
                                                           &                                                                    & Forgot to add the \lstinline!repeat! keyword                                                             \\ \hline
\end{tabular}
\caption{The table of identified problems categorised by severity}
\end{table}

\textbf{Critical problems}
\begin{enumerate}
\item \textbf{Not using the \lstinline!end! keyword at all} - this would affect the overall validity of the program because the scoping rules in Quorum are defined in conjunction with the \lstinline!end! keyword. This shows a fundamental misunderstanding of how scoping works in the language
\item \textbf{The lack of constructors with parameters in Quorum} - Quorum does not support constructors with parameters which might be problematic for the participants, having experience with other languages where this feature is common. It both causes the participants to invoke syntax in the class that is not supported, and have difficulties instantiating classes. Since this is a significant difference in how the code should be structured, it is considered critical.
\item \textbf{Misunderstanding the effect of \lstinline!Sort()! on arrays of objects} - The inbuilt sorting function for arrays does not have access to the properties of the objects and therefore does not sort them by any of those. This would have the consequence of code, written with the assumption that it works, be most likely wrong, which means recovery would require a full rewrite of the algorithm. This makes the problem critical. It is possible that with the use of a compiler the participant would discover and recover much easier, which could mean the problem would potentially be considered serious.
\end{enumerate}

\textbf{Serious problems}
\begin{enumerate}
\item \textbf{Not using the \lstinline!end! keyword to end the scope of if-statements} - Although this problem looks similar to the first problem defined in Critical problems, the difference is that it is more likely to be an overlook than a misunderstanding of the scoping rules in Quorum. Also single-line if-statements might be present in other languages and not in Quorum.\\
\item \textbf{Forgetting to increment the iterator in a repeat while loop} - This could be considered an oversight on the participant’s part, attributed to how the repeat-while construct works in Quorum, compared to how usually for-loops are used, and therefore it was not critical. However, it is still considered a serious problem because of the impact it has on the structural correctness of the code.
\item \textbf{The lack of common looping constructs (for-loops or foreach loops)} - This is considered a serious problem for few reasons. Firstly, it warrants the use of the repeat-while construct as a part of Quorum, which might not be so intuitive for people coming with backgrounds in other languages, where these constructs are present. Secondly, this might compound to the previous problem described in this section which would have a high impact on the validity of the written program. 
\item \textbf{Forgetting to import a library for containers (array)} - Containers in Quorum, and specifically arrays, have to imported first before being used. This is considered a serious problem since it might have a high impact on the validity of the program. 
\item \textbf{Not using \lstinline!elseif! to avoid having to close an additional scope} - This problem is serious because it shows a lack of understanding the finer points of scoping in an if-else chain. This is problematic since \lstinline!else if! is also a valid syntax, but carries unintended consequences.
\item \textbf{Not resetting inner loop iterator between loops} - Similarly to serious problem 2, this could be considered an oversight on the participant’s end due to previous experience with other programming languages.
\end{enumerate}

\textbf{Cosmetic problems}
\begin{enumerate}
\item \textbf{Using colon (\lstinline!:!) instead of dot (\lstinline!.!)} - This is considered a cosmetic problem since does not affect the structure of the program being only an exchange of a single character. It could be said that most of the participants had a programming bias given their background in other programming languages where the dot notation is common.
\item \textbf{The lack of aggregate operators in Quorum (e.g.\lstinline!+=!)} - This is considered a cosmetic problem since it does not affect the correctness of the program but it is rather a matter of convenience for the participants.
\item \textbf{Using conditional AND and OR as \lstinline!&&! and \lstinline!||! instead of \lstinline!and! and \lstinline!or! keywords, as defined in Quorum}  - This problem is considered cosmetic because the participants did not use the correct keywords in the context but had the proper intentions to do so. This could be attributed to the simple matter of not properly reading the sample sheet to find the proper keywords and using the ones they know from other languages instead.
\item \textbf{Using the \lstinline!float! instead of \lstinline!number! keyword} - This is considered a cosmetic problem because it does not have a big impact on the program’s correctness but rather is using a naming convention from other programming languages
\item \textbf{Using \lstinline!string! instead of the \lstinline!text! keyword} - this is the same as with the previous cosmetic problem
\item \textbf{Writing \lstinline!output! instead of \lstinline!return! as the keyword for a return statement} - This is cosmetic because it is mostly a result of our sample sheet using \lstinline!output! often, while the participants were more commonly expected to write code returning something and had a familiarity with the \lstinline!return! keyword from other languages.
\item \textbf{Using \lstinline!==! in conditional statements instead of \lstinline!=!} - Similarly to previous cosmetic problems, the main reason behind this problem is that most of the participants had experience with other programming languages where the \lstinline!==! notation is common and in turn had a particular bias against using the \lstinline!=! notation.
\item \textbf{Using \lstinline!int! instead of \lstinline!integer!}  - same as with cosmetic problem 4
\item \textbf{Using \lstinline!bool! instead of \lstinline!boolean!} - same as with cosmetic problem 4
\item \textbf{Typos in library importing} - This is a simple case of having small typos when writing the import code. Easily fixed and a cosmetic problem.
\item \textbf{Mistyping \lstinline!integer! as \lstinline!integar!}  - this is a cosmetic problem since it is a simple typing mistake and it does not have any impact on the validity of the program.
\item \textbf{ Accidentally used \lstinline!0! instead of \lstinline!O! in variable name}  - Again another small typo and therefore cosmetic.
\item \textbf{Mistyped the \lstinline!is! keyword as \lstinline!ia!}  - similar to cosmetic problem 12
\item \textbf{Forgot to add the \lstinline!repeat! keyword} - This problem is considered cosmetic since it does not have a significant impact on the correctness of the program.
\end{enumerate}

\section{Comparison with Quorum's evidence}
In their empirical experiments, Stefik and Gellenbeck \cite{EmpStudiesonStimuli} gathered many statistically significant results regarding keyword choices. Given that they try to primarily address visually impaired people and novice programmers, selecting the most intuitive words seems like a logical choice. For one of their  experiments \cite{EmpStudiesonStimuli}, they divided the participants in two groups - novices and experienced programmers to find out if there is a significant discrepancy in the results between the two groups. Every keyword choice was ranked in two tables by mean value and standard deviation. Since we conducted our experiment with people having programming experience, we are primarily interested in the results of the second group. For the purpose of our evaluation method, we will not mention every single word choice they rated but rather the ones which are coinciding with the problems we identified from the IDA evaluation in section \ref{section: Problem Categorization}. Additionally, we would also relate two of the empirical studies from Stefik and Siebert \cite{Empiricalinvestigation} and their findings about keyword choices. This would help us to make a comparison between the authors findings and the results from our evaluation, essentially giving an additional degree of credibility to the method if similarities are found.

\begin{itemize}
\item In the results for the concepts of AND, OR and XOR, Stefik and Gellenbeck \cite{EmpStudiesonStimuli} found that for the AND concept, using \lstinline!&&! and \lstinline!and! performed quite well. Their results showed that these words are actually popular and thus intuitive to use. As for the logical OR concept, the \lstinline!or! keyword was placed first, being significantly better the the second highest one - the \lstinline!||! operator, which is present in many popular programming languages. Last but not least, the XOR logical operator, the \lstinline!or! was rated highest which can be attributed to the fact that the participants did not know how to call an operation which "took a behavior when one condition was true but not both".
\item Stefik and Gellenbeck \cite{EmpStudiesonStimuli} settled on using a single equals (\lstinline!=!) sign for assignment statements and for testing equality since that is what they thought would make most sense. Although this might be true for the novice group (single equals (\lstinline!=!) sign was ranked highest), the experienced programmers group did not even rate the single equals in the top 3, rating the double equals (\lstinline!==!) as highest instead. This was not verified until one of the later empirical studies by Stefik and Siebert \cite{Empiricalinvestigation}.
\item For the concept of "Taking a behaviour" the authors considered several word choices such as \lstinline!function!, \lstinline!action! and \lstinline!method!. The novices ranked the \lstinline!action! word the highest, while the experienced programmers - \lstinline!operation!, followed by \lstinline!action!, \lstinline!method! and \lstinline!function!. However, the authors admit that this particular results should be further investigated, since the participants might have understood the description of the concept as something other than completely capturing the idea of a function.
\item Quorum makes use of the keyword \lstinline!repeat! over \lstinline!for!, \lstinline!while! or \lstinline!cycle! (see Sanchez and Flores \cite{SanchezData}) following a study which shows that \lstinline!repeat! represents the concept of iteration significantly better than the the aforementioned words \cite{EmpStudiesonStimuli}.
\end{itemize}

Based on the results from Stefik and Siebert \cite{Empiricalinvestigation}, the programmers group found \lstinline!==! to be intuitive as the boolean equals operator, which matches our observation of the participants often using \lstinline!==! instead of \lstinline!=! notation. For many of our other problems where our participants used the wrong syntax, Stefik and Siebert\cite{Empiricalinvestigation} did have comparable results were both the wrong and the correct syntax was found intuitive by their programmers group. These are: 
\begin{itemize}
\item Dot (\lstinline!.!) versus colon (\lstinline!:!)
\item using an aggregate operator (\lstinline!x += 1!) versus an arithmetic operator (\lstinline!x = x + 1!)
\item \lstinline!&&! versus \lstinline!and! for logical AND
\item \lstinline!||! versus \lstinline!or! for logical OR
\item data types wording (\lstinline!float! versus \lstinline!number!, \lstinline!string! versus \lstinline!text! and \lstinline!bool! versus \lstinline!boolean!)
\end{itemize}

It is possible that a lot of these cases were the result of a participant just glancing over the sample sheet, instead of having a more thorough look at the syntax on the sample sheet. Since the constructs looked intuitive, they did not notice or remember that it was different and thus just used the syntax they were used to from other programming languages. Interestingly, our results about the looping constructs contradicts the results from \cite{Empiricalinvestigation}. Our participants often lamented the lack of a \textit{for} or \textit{foreach} loop and had a lot of errors in using the iterator for the \textit{while} loop. This is in contrast to the papers results where their programmers did not find \lstinline!for! among the most intuitive and found \lstinline!foreach! among the least intuitive keywords for looping. However, the results are not directly contradictory as the paper's questions about intuitiveness was focused on the syntax, while our participants problems were more about lacking the functionality of a looping construct with inbuilt iterator handling. A more direct contradiction is that our participants often found the \lstinline!repeat! keyword unnecessary, despite the paper listing it as one of the most intuitive keywords for looping. This could be a side effect of us only demonstrating the \textit{repeat while} loop in our sample sheet, since that loop looks exactly like the \textit{while} loop they are used to but with an extra keyword in front. 

A lot of our results were unsurprising.
Quorum is a language that uses evidence about programming to design a language that is intuitive for novices.
Since our participants were experienced programmers though, it would be expected that a lot of the errors encountered would be related to this mismatch.
Especially the critical error with lacking constructors with parameters, showed a large mismatch in what an experienced programmer expected from a class compared to what was proven to be more user-friendly\cite{ParamConstructors}.
Likewise the lack of a for- or foreach loop and the resulting iterator handling problems experienced by our participants, showed that they had a habit of handling the iterator in the looping construct.
This functionality, however, could make the construct less practical for novices, as they might get a better understanding of the same functionality by writing the statements separately.
More surprisingly we had a participant who never used the \lstinline!end! keyword.
In the discussion he explained this was because he thought indentation was used to control scope.
He felt that since indentation is a good practice that all programmers should use anyway, it would make sense to make the language use and enforce this.
This would be especially true for a beginner language, as the beginners are those who need to learn to use indentation.
This again showed that experienced programmers were likely to draw from their previous experiences rather than thoroughly examine the sample sheet.

\section{Interview Results}
During the interview we collected feedback addressing various key points such as the usability of programming languages (in particular Quorum), programming without a compiler or the help of an IDE, impressions of how effective some constructs are and how they can be improved. The most common observations among the participants were:

\begin{description}
\item[The use of Quorum as a programming language] - the majority of the participants (\#3, \#4, \#5, \#8) found Quorum an easy to use and understand language. Additionally, some compared it and found it similar to other languages such as C, C\#, Pascal and Python and generally less verbose than standard OO languages they had experience with (e.g. C\#,Java). 
\item[Managing scoping rules by the use of the \lstinline!end! keyword instead of brackets] - Generally, most of the participants found the use of the \lstinline!end! keyword for defining scopes very confusing. Participants \#4 and \#6 preferred the use of brackets, similar to OO languages they were familiar with (e.g. Java and C\#), while participants \#7 and \#8 preferred indentation similar to languages like Python. Participant \#7 further suggested to extend the \textit{end} construct to \textit{begin-end}, similar to Pascal which would make the language more user-friendly for novices. 
\item[Quorum uses the colon (\lstinline!:!) notation instead of dot(\lstinline!.!)] - Given their prior experience with programming languages where the dot (\lstinline!.!) notation is common, participants \#4, \#6 and \#7 found it confusing to use the  colon (\lstinline!:!) notation instead. This confusion was further reinforced by the fact that the dot notation is still used when calling libraries.
\item[The lack of common control statements such as for and for-each loops was confusing] - The lack of common control statements such as \textit{for} or \textit{for-each} loops in Quorum seemed like a hurdle for the participants and consequently they found it not so intuitive to use the \textit{repeat while} construct as a substitute of that. This is evident by the fact that some of them completely forgot to include it in the loop's signature (participant \#3) or found it unnecessary altogether (participant \#7).
\end{description}
\chapter{Discussion}
The results from Chapter \ref{chapter:Results} address some potential problems when working with a language such as Quorum. This section will elaborate on how that can be extended to other programming languages and how our method could be used in a customized manner.
\\
A lot of our results were unsurprising.
Quorum is a language that uses evidence about programming to design a language that is intuitive for novices.
Since our participants were experienced programmers though, it would be expected that a lot of the errors encountered would be related to this mismatch.
Especially the critical error with lacking constructors with parameters, showed a large mismatch in what an experienced programmer expected from a class compared to what was proven to be more user-friendly for a novice\todo{ref usability study of constructors}.
\todo{maybe start here instead as the above might be better for the comparison in results}
Comparing our results with Quorum's evidence has shown that our method gets comparable results to other methods but with a significantly lower amount of participants.
Most of Quorum's evidence about experienced programmers has however been focused on just the syntax.
This means that most of the comparable data lies in our cosmetic errors, which are usually the least interesting problems from a usability standpoint.
The more serious problems tend to either contradict or be not addressed by Quorum's evidence, though in most cases this is a result of the mismatch in target group between novice and experienced programmers.
One noticeable problem we encountered in the execution of our test was participants freezing at the very beginning.
They were unsure how to start as they were could not figure out what format of the solution they should use.
For most of the participants this was not a big hurdle as they would either just pick one way of doing it or consult the facilitator.
However for some participants, having a discussion while programming was unnatural.
One way to prevent this could be to have some pre-written code that the participant should instead fill out.
this also would provide a more structured presentation of the task, which can help create a more uniform scope in the solution between participants\todo{have we highlighted the troublesome task scoping yet?}.
It does however sacrifice some of the potential data about the language that a more free-form task can give.

\section{Threats of validity}