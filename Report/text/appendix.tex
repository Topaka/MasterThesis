\chapter{Interview notes}
\label{chapter:Interview notes}
As previously mentioned, the interviews with the participants were recorded in order to preserve the necessary feedback which helped in analysing the results. Instead of providing the entirely of the interviews in the form of transcripts, we decided to condense the information in key points instead. This helped us to analyse the data from the interviews much easier and find out how many occurrences of a given problem there are across all the participants. Additionally, this section encapsulates the essential parts of each interview and highlights what every participant had to say in terms of feedback, suggestions for further improvements and encountered problems.
\section{Participant {\#}1 (Pilot Test)}
\begin{itemize}
\item Thought that using colon (“:”) instead of dot (“.”) was weird, both because it goes against the norm (the participant had experience with several languages which use the dot notation) and because dot is easier to type. 
\item Thought that tasks are trivial to understand but take time to code
\item Task 1 was too broad in the definition causing the task to be too large and time-consuming and the participant to spend time on unintended things.
\item Mentioned that although repetitive to a certain extend, task 2 was tricky and very good at conveying that you have to pay attention when copying code. 
(*He actually fell into that trap and he did not realise it up until the facilitator intervened and pointed that out.) 
\item Thought that tasks 2 and 4 were quite good in terms of their intended purposes while task 3 (operation on strings) was trivial and very similar to task 2. 
\item He found the samples of conditional use not being able to clearly convey the differences between Quorum and other known languages he had experience in. In particular "==" vs. "=" and "and" vs "\&\&" did not stand out.
\item Thought that it was good that the sample sheet was split in categories to make it easier for the facilitator to reference them when asked.
\end{itemize}
\section{Participant {\#}2}
\begin{itemize}
\item Thought that a lot of the notations were unintuitive because they differed from the mathematical norm
\item Found the keyword "action" confusing 
\item Thought that using "loop" would be easier than "repeat"
\item It was difficult for him to devise the code needed for solving the tasks, although he found the mathematics behind quite easy
\item It was daunting to not have any fallback or assistance when trying to code and learn how to code without a compiler
\end{itemize}
\section{Participant {\#}3}
\begin{itemize}
\item Suggested that we should add specific values for task 3
\item Wondered how to define return types of an action
\item Quorum does not have parameterized constructors
\item Suggested that we add how to get the size of an array with an in-build action
\item Forgot to increment loop counters
\item Forgot to add the “repeat” keyword
\item Thought that Quorum has a limited number of looping constructs, but it is easy to learn, write and understand
\item Quorum is very terse
\item Thought that “output” makes more sense than using “print”
\item Thought that “returns” of an action seems intuitive
\item Liked the “is” keyword for class inheritance
\item Thought that Quorum is similar to C, with a different syntax (programmer-friendly C) 
\item Thought that the lack of parameterized constructors is not that limiting, but does not have enough experience to say with a certainty
\item Found the tasks not too challenging 
\item Thought that not using a compiler is not much of a hindrance
\item Found the example sheet indispensable and very helpful
\item Suggested that we could add more examples for loping constructs
\end{itemize}
\section{Participant {\#}4}
\begin{itemize}
\item Found it strange to use words as a means of closing scopes instead of brackets as well as using colons instead of dots
\item Thought the languages is straightforward and easy to use
\item Used a “float” instead of a “number” keyword, as well as “string” instead of “text”
\item Forgot to add “returns” keyword at the end of an action
\item Forgot to increment the counters on loops
\item Had some problems with scoping by making use of the “end” keyword
\item Found the tasks specific, understandable and clear
\item Thought that “.” makes more sense than “:”
\item Suggested that we add an example of method inheritance on the example sheet
\item Suggested that we change the “/” on task 2 with an “or”
\item Suggested that we add a sort action on the example sheet
 constructs
\end{itemize}
\section{Participant {\#}5}
\begin{itemize}
\item Found Quorum is similar to C
\item Quorum has similar design to other languages “string” instead of “text”
\item Tended to over-complicate things and thus - over-engineer the tasks
\item Made use of the example sheet quite frequently
\item Coding without a compiler was unpleasant and felt like being in an exam, unable to get a feedback from what’s being written down (Does not allow a great deal of experimentation)
\item Had difficulties with the syntax of arrays - using the “[]” notation instead of the get(i) inbuilt method
\item Forgot to write the import for using arrays, as specified on the example sheet
\item Found the tasks very good at conveying our intended purposes and easy to understand
\item Found the amount of tasks good and reasonable
\item Found Task 3 to be a bit tricky since you have to specifically think in terms of inheritance from the start
\item Found the example sheet informative and referred to it several times
\end{itemize}
\section{Participant {\#}6}
\begin{itemize}
\item Quorum’s design  seems a bit confusing
	\begin{itemize}
		\item Closing the scopes of If-statements with “end”
		\item Lack of parameterized constructors
		\item Lack of a for-loop
	\end{itemize}
\item Found the tasks very good and the example sheet - very concise
\item Found coding without a compiler scary without “the safety net”
\item Typed “=” instead of “==” for an inequality operator
\item Thought it might be more intuitive to use a Get method directly compared to how it is being used in the language
\item Typed “.” instead of “:”
\item Thought that ending classes with something different than the “end” keyword will make more sense
\item Found closing the scope of if-statements with the “end” keyword confusing and said that brackets would make it more readable (similarly to OO languages such as Java and C\#)
\item Found Quorum less verbose than other OO languages
\item Forgot to increment the counter variable outside of a loop
\item Although the participant over complicated the tasks based on the provided description, he found them very good and efficient at what they try to convey
	\begin{itemize}
		\item Task 2 - the description of the task seems rather confusing, which made the participant to over-engineer the solution
		\item Task 3 - doable 
	\end{itemize}
\item Found the example sheet contains enough content in order to solve the tasks  
\item Had a few suggestions how to improve the overall look of the example sheet
\end{itemize}
\section{Participant {\#}7}
\begin{itemize}
\item Found Quorum similar to Pascal and C\#
\item Liked certain parts of the language and disliked others
\item Found the use of “repeat” unnecessary since it does not make sense in conjunction with the standard loop wording
\item Noticed that you have to close a class/action with an “end” keyword
\item Suggested that implicit type casting would be better for novices
\item “:” used in different scenarios might be confusing
\item Thought that the “returns” keyword can have a better placement in the action’s signature
\item Noticed that you have to use a library for an array
\item Said that the “end” keyword does not make much sense and rather see a “begin-end” scoping construct, similar to Pascal and Python - only indentation
\item Casting data types could be dangerous for novices
\item Found the “returns” keyword’s placement not so intuitive 
\item Found the “end” keyword for if-statements not so adequate, can use indentation instead similar to Python
\item Found the tasks very good:
	\begin{itemize}
		\item Task 2’s challenge of reusing code is a good exercise 
		\item Task 2 could have a 2 predefined lists with names
	\end{itemize}
\item Said that the task encompass a good portion of constructs 
\item Suggested we could add a setup for easier start with the tasks
\item Suggested we give better titles on the examples sheet and better indexing when looking for things
\item Coding without a compiler did not matter that much in his opinion
\item Found it great that the facilitator could say if the task is done or not
\item GetSize() and Add() in-build methods examples were missing
\item Acknowledged that the code samples are highlighted and there are working examples
\item Said that we should be consistent with the working titles
\end{itemize}
\section{Participant {\#}8}
\begin{itemize}
\item Found Quorum intuitive to use, but limited in terms of available constructs
\item Suggested that “returns nothing” would be intuitive
\item Found the naming of keywords inconsistent (Arrays with capital A and everything else with small letters)
\item Found it confusing not to use indentation for scopes
\item Found the lack of semicolons a very good thing
\item Liked the “is” keyword for class inheritance
\item Pointed out the lack of a “continue” construct for loops
\item Would have liked more features from functional programming
\item Suggested we could make the “or” and “and” statements bolded in task 2
\item Noticed the lack of an aggregate “+= “ operator
\item Quorum reminds him of OO languages and similar to Python
\item Would have liked a summary of all the examples on the examples sheet
\item Found the examples not so sufficient per task
\item Suggested that we could highlight important parts on the task sheet 
\item Found the lack of a compiler while coding “dangerously scary”
\item Over-engineered task 1 
\item Suggested that we could have an additional sheet with solutions to the tasks
\item Separate each task on a separate paper so it is easier to navigate 
\end{itemize}
\chapter{Participant code}
\section{Participant \#1}
\begin{lstlisting}[language=Quorum,tabsize=2]
use Libraies.Container.Array
use public 

action Main
Array<Cucumber> a 
a:add(Cucumber c1)
	a:add(Cucumber c2)
	a:add(Cucumber c3)
	a:add(Cucumber c4)
CalcTotal(a)
end


action CalcTotal(Array<Cucumber> arr)
	number total = 0
	integer i = 0
	total = Cucumber.Price(arr.GetSize())
	output "total = " + total
	
end

Class Cucumber
	integer id
	number price
	number bulkPrice
	integer bulkCount
	number percentageDiscount
	boolean bORp
	
	
	public action Price(integer amount)
		integer remains = mod amount
		integer numdiscount = amount / bulkCount
		
		number value
		if bORp
			value = remains* price + numdiscount * bulkPrice
		else
			value = 100 - percentageDiscount * price *amount
		end
		
		return value
	end
end


//TASK2

Class Player 
	public text FN
	public text LN
	public integer age
	
	action make (text first, text last, integer _age)
	FN = first
	LN = last
	age = _age
	end
end

action Main
Aray <Player> T1
Array <Player > T2

Player p1
Player p2
Player p3

Player p4
Player p5
Player p6

p1:make("a","b",10)
p2:make("a",zebra,1)
p3:make("gi", "joe", 65)

T1:add(p1)
T1:add(p2)
T1:add(p3)

p4:make("anotherguy","b",20)
p5:make("c","c",11)
p6:make("d","d",25)

T2:Add(p4)
T2:Add(p5)
T2:Add(p6)

	public action FindFFNLNbetweenTeams returns integer
		integer i = 0
		integer j = 0
		integer found = 0
		repeat while i < T1:GetSize()
			repeat while j < T2:GetSize()
				 if T2:Get(j):FN = T1:Get(i):FN 
					found = found +1
				else if T2:Get(j):LN = T1:Get(i):LN 
					found = found +1
				end
			end
		end
		return found
	end
	
		public action FindFFNLNinTeam(Array<Player>) returns integer
		integer i = 0
		integer j = 0
		integer found = 0
		repeat while i < T:GetSize()
			repeat while j < T:GetSize()
				 if T:Get(j):FN = T:Get(i):FN 
					found = found +1
				else if T:Get(j):LN = T:Get(i):LN 
					found = found +1
				end
			end
		end
		return found
	end

		public action FindAgebetweenTeams returns integer
		integer i = 0
		integer j = 0
		integer found = 0
		repeat while i < T1:GetSize()
			repeat while j < T2:GetSize()
				 if T2:Get(j):Age = T1:Get(i):Age 
					found = found +1
				end
			end
		end
		return found
	end
	
		public action FindAgeinTeam(Array<Player>) returns integer
		integer i = 0
		integer j = 0
		integer found = 0
		repeat while i < T:GetSize()
			repeat while j < T:GetSize()
				 if T:Get(j):Age = T:Get(i):Age 
					found = found +1
				end
			end
		end
		return found
	end
	
	
		public action FindAgeinTeam(Array<Player>) returns integer
		integer i = 0
		integer j = 0
		integer found = 0
		repeat while i < T:GetSize()
			repeat while j < T:GetSize()
				 if T:Get(j):Age = T:Get(i):Age and T:Get(j):FN = T:Get(i):FN
					found = found +1
					j = j+1
				end
				i = i+1;
			end
		end
		return found
	end
	
end


//TASK 4

Class Base

integer hp
integer dmg

	action do()
	end
	action Attack(Base target )
	target:takeDamage(dmg)
	end

	action takeDamage(integer damage)
	hp = hp - damage
	  if hp >= 0 
	    kill()
	  end
	end
	
	action replnishHP(integer amount)
	hp = hp + amount
	end
	
	action kill ()
	delete me
	end
	
end

Class Warior ia Base
hp = 150
dmg = 10;
integer fury = 100
	
	action do()
	fury = fury +1
	end
	
	action strongAttack
		if fury > 10
		  target:takeDamage(dmg+10)
		  fury = fury - 10
		else
		output "might knight whines like tiny baby men"
		end
	end
end

Class Mage is Base
hp = 70
damage = 12
integer mana

	action do()
	mana = mana +1
	end
	

	action heal (target)
		if mana > dmg
		target.replnishHP(dmg)
		mana = mana - dmg
		end
	end
	
end

Action Main
for each base
do()
end
end
\end{lstlisting}

\section{Participant \#2}
\begin{lstlisting}[language=Quorum,tabsize=2]
action gettotal (integer Oranges,integer Bananas,boolean isregular) returns integer
integer total=0
total=total+Oranges*5+Bananas*4-5*Oranges/3-10*Bananas/5
if isregular = true
total=total*0.9
end
refurn total
end
output gettotal (3,5,true)

first name  last name  age  team
\end{lstlisting}

\section{Participant \#3}
\begin{lstlisting}[language=Quorum,tabsize=2]
class Test1
	integer OrangePrice = 5
	integer BananaPrice = 4
	
	action TotalPrice(integer oranges, integer bananas)
		Output oranges * Orangeprice + bananas * BananaPrice
	end
	
	action TotalPriceWithDiscount(integer oranges, integer bananas, boolean regular)
		number result = 0
		repeat while oranges > 3
			result = result + 10
			oranges = oranges - 3
		end
		result = result + oranges * OrangePrice
		
		repeat while bananas > 5
			result = result + 10
			bananas = bananas - 5
		end
		result = result + bananas * BananaPrice
		
		if regular
			Output result * 0.9
		else
			Output result
		end
	end
end

class Test2
	Array<Player> Team1
	Array<Player> Team2
	
	action SameFirstLastNameSameTeam(Array<Player> team) returns boolean
		integer i = 0
		repeat while i < team.GetSize()
			text firstName = team:Get(i):FirstName
			text lastName = team:Get(i):LastName
			integer j = 0
			repeat while j < team.GetSize()
				if not(j == i) and (firstName == team:Get(i):FirstName or lastName == team:Get(i):LastName)
					return true
				end
			end
		end
		return false
	end
	
	action SameFirstLastNameDifferentTeams() returns boolean
		integer i = 0
		repeat while i < Team1.GetSize()
			text firstName = Team1:Get(i):FirstName
			text lastName = Team1:Get(i):LastName
			integer j = 0
			repeat while j < Team2.GetSize()
				if firstName == Team2:Get(i):FirstName or lastName == Team2:Get(i):LastName
					return true
				end
			end
		end
		return false
	end
	
	action SameFirstLastNameSameTeam(Array<Player> team) returns boolean
		integer i = 0
		repeat while i < team.GetSize()
			text firstName = team:Get(i):FirstName
			integer age = team:Get(i):Age
			integer j = 0
			repeat while j < team.GetSize()
				if not(j == i) and (firstName == team:Get(i):FirstName and age == team:Get(i):Age)
					return true
				end
			end
		end
		return false
	end
end

class Player
	text FirstName
	text LastName
	integer Age
end

class Character
	number Health

	action ReplenishHealth(integer amount)
		Health = Health + amount
	end
		
	action Attack(Character target)
		target:Health = target:Health - 10
	end
end

class Mage is Character
	number Mana
	
	action Fireball(Character target)
		if Mana >= 10
			Mana = Mana - 10
			target:Health = target:Health - 20
		end
	end
	
	action ReplenishMana(integer amoutn)
		Mana = Mana + amount
	end
end

class Warrior is Character
	number Fury
	
	action Execute(Character target)
		if Fury >= 25
			Fury = Fury - 25
			if target:Health < 30
				target:Health = 0
			else
				target:Health = Target:Health - 10
			end
		end
	end
	
	action ReplenishFury(integer amount)
		Fury = Fury + amount
	end
end

class Test4
	text Text = "Rasmus Moeller Jensen"
	Array<Text> a = Text:Split("")
	
	action PrintReverse()
		integer i = a:GetSize() - 1
		text Result = ""
		while i >= 0
			Result = Result + a:Get(i)
			i = i + 1
		end
		Output Result
	end
	
	action PrintAlphabetical()
		Array<Text> b = a
		b:Sort()
		text Result = ""
		integer i = 0
		repeat while i < b:GetSize()
			Result = Result + b:Get(i)
			i = i + 1
		end
		Output Result
	end
	
	action FindDuplicates()
		integer i = 0
		integer j = 0
		integer Result = 0
		Array<Text> AlreadyTested
		
		repeat while i < a:GetSize()
			j = 0
			repeat while j < a:GetSize()
				if not(i == j) and not(a:Get(i) == " ")	and not(a:Get(j) == " ") and not(AlreadyTested:Contains(a:Get(i))) and a:Get(i) == a:Get(j)
					Result = Result + 1
					AlreadyTested:Add(a:Get(i))
				end
				j = j + 1
			end
			i = i + 1
		end
		Output Result
	end
end
\end{lstlisting}

\section{Participant \#4}
\begin{lstlisting}[language=Quorum,tabsize=2]
use Libraries.Containers.Array


action CalculatePrice(integar nBananas, number pBananas, integar nOranges, number pOranges, boolean regular) returns number
	number totalgroup = ((nBananas / 5) * 10) + ((nOranges / 3) * 10)
	number rBananasPrice = (nBananas % 5) * 4
	number rOrangesPrice = (nOranges % 3) * 5
	number total = totalgroup + rBananasPrice + rOrangesPrice
	if regular total = total * 0.9
	return total
end

//each array entry is a string with name, second name
action FindSameFirstNames(Array<text> teamone, Array<text> teamtwo) returns string
	Array<text> SameNames;
	integar i = 0
	integar j = 0
	repeat while i < teamone:GetSize()
		repeat while j < teamtwo:GetSize()
			string pAF = teamone.Get(i).Split(",").Get(0)
			string pBF = teamtwo.Get(j).Split(",").Get(0)
			string pAL = teamone.Get(i).Split(",").Get(1).Trim()
			string pBL = teamtwo.Get(j).Split(",").Get(1).Trim()
			if(pAL = pBL or pAF = pBF) return players.Get(i) + " : " + players.Get(j)
			i = i + 1
			j = j + 1
		end
	end
end

//each array entry is a string with name, second name, age
action FindSameFirstNamesAndAge(Array<text> teamone) returns string
	Array<text> SameNames;
	integar i = 0
	repeat while i < teamone:GetSize()
		integar j = 0
		repeat while j < teamone:GetSize()
			string pAF = teamone.Get(i).Split(",").Get(0)
			string pBF = teamone.Get(j).Split(",").Get(0)
			integar pAA = cast (integar, teamone.Get(i).Split(",").Get(2))
			integar pBA = cast (integar, teamone.Get(j).Split(",").Get(2))
			if(pAF = pBF and pAA = pBA) return players.Get(i) + " : " + players.Get(j)
			j = j + 1
		end
		i = i + 10
	end
end



class character
	public integar hp = 100
	public integar resourceAmount = 100
	
	public action Attack(character defender, integar amount)
		defender:hp = defender:hp - amount
	end
	
	public action Recover(integar amount)
		hp = hp + amount
	end
	
	public action RecoverResource(integar amount)
		resourceAmount = resourceAmount + amount
	end
end

class mage is character

	public string resourceName = Mana
	
	public action Fireball(character defender, integar amount)
		parent:character:Attack(defender,amount)
		parent:character:resourceAmount = parent:character:resourceAmount - 10
	end
	
end

class warrior is character

	public string resourceName = Rage 
	
	public action Pummel(character defender, integar amount)
		parent:character:Attack(defender,amount)
		parent:character:resourceAmount = parent:character:resourceAmount - 20
	end
	
end
	
class taxAccountant is character

	public string resourceName = Money
	
	public action ChargeWithTaxEvation(character defender, integar amount)
		parent:character:Attack(defender,amount)
		parent:character:resourceAmount = parent:character:resourceAmount - 50
	end

end

public action ReverseText(text texttotreverse) returns text
	text out = ""
	Array<text> characters = texttotreverse:Split("")
	integar count = character:GetSize() - 1
	repeat while count >= 0
		out = out + characters:Get(count)
		count = count - 1
	end
	return out
end

public action SortCharacters(text string) 
	Array<Text> characters = string:Split(""):Sort()
	integar count = character:GetSize()
	integar i = 0
	repeat while i < count 
		output characters:Get(i)
		i = i + 1
	end
end

public action FindDuplicates(text string) returns integar

	Array<text> characters = string:Split("")
	Array<text> foundChar
	
	integar i = 0 
	
	repeat while i < characters:GetSize()
		integar j = 0
		repeat while j < characters:GetSize()
			if characters:Get(i) = characters:Get(j)
				integar k = 0
				boolean found = false
				repeat while k < foundChar:GetSize()
					if characters:Get(i) = foundChar:Get(k)
						found = true
				end
				if not found
					foundChar:Add(characters:Get(i)
				
			j = j + 1
		end
		i = i + 1
	end
	
	return foundChar:GetSize()
end
\end{lstlisting}

\section{Participant \#5}
\begin{lstlisting}[language=Quorum]
action Main
	action calculateFruit(integer banana, integet orange) returns integer
		integer orangePrice = 5
		integer bananaPrice = 4
		
		return orange*orangePrice + banana*bananaPrice
	end
	
	action calculateFruit(integer banana, integet orange, boolean regular) returns integer
		integer orangePrice = 5
		integer bananaPrice = 4
		
		orangesDiscount = orange/3
		orangeRemainder = orange mod 3
		bananaDiscount = banana/5
		bananaRemainder = banana mod 3
		
		sum = orangeRemainder*orangePrice + orangesDiscount*10 + bananaRemainder*bananaPrice + bananaDiscount*10
		
		if regular
			return sum-sum*0.1
		else
			return sum
		end
	end
	
	// firstname,lastnam
	
	// 0,firstname
	// 1,lastnam
	action findPlayers1(Array<text> team) returns Array<text>
		
		integer i = 0
		Array<Array<text>> players
		repeat while i < team:GetSize()
			Array<text> player = team:Get(i).Split(",")
			players:Add(player)
			i = i + 1
		end
		
		integer i = 0
		integer j = 0
		repeat while i < players:GetSize
			repeat while j < players:GetSize
			players:Get(i):Get(0) == players:Get(j):Get()
			
			
		
	end
	
	class Warrior is character
		
	end
	
	class Mage is character
	
	end
	
	class character
		integer hitPoints
		public action attack(character c)
		
			character:decreaseHitpoint(200)
		end
		
		public decreaseHitpoint(integer amount)
			hitPoints = hitPoints - amount
end
\end{lstlisting}

\section{Participant \#6}
\begin{lstlisting}[language=Quorum,tabsize=2]
action main
	integer sum = 0
	Array <Product> prod = basket:Get()
	integer count = 0
	repeat while count<prod:GetSize()
		sum  = sum + prod:GetProd():GetPrise()
	end
	
	integer count2 = 0
	
	repeate while count< prod:GetSize()
	if (prod:GetProd == 'oranges' )
		integer numOfOrn = numOfOrn + 1
	else 
		integer numOfBan = numOfBan + 1
	end
	
		integer tripOrn = numOfOrn / 3
		integer discountprice = tripOrn  * 10
		integer normalprice = (numOfOrn - tripOrn) * 15
		integer totalpriceOrn = discountprice + normalprice
		
		integer fiveBan = numOfBan / 5
	
	if basket:GetCustomer():IsRegular == true
		discountprice = price * 0.9
		
end


Task 2

action Main

	Array <Player> team1 = GetTeamPlayers()
	Array <Player> team2 = GetTeamPlayers()
	team1:Sort()
	team2:Sort()
	
	
	integer i = 0
	repeat while i < team2:GetSize()
		if team1:GetPlayer(i):GetPlayerFName() = team1:Get(i+1):GetPlayerFName or team1:Get(i):GetPlayerLName() = team1:Get(i+1):GetPlayerLName
		output  `Same team : ' + team1:Get(i):GetPlayerFName  +  team1:Get(i+1).GetPlayerFName
		
		else if team1:Get(i):GetPlayerFName() = team2:Get(i):GetPlayerFName or team1:Get(i):GetPlayerLName() = team2:Get(i):GetPlayerLName
		output "different teams :" + team1:Get(i):GetPlayerFName  +  team1:Get(i+1).GetPlayerFName
		
		else team1:GetPlayer(i):GetPlayerFName() = team1:Get(i+1):GetPlayerFName or team1:Get(i):GetPlayerAge() = team1:Get(i+1):GetPlayerAge
		
		output  'Same team : ' + team1:Get(i):GetPlayerFName  +  team1:Get(i+1).GetPlayerFName + "Same age"
		end
	end	
end

Task 3

class StartGame
	action Main
	
	end
end

class Hero
	integer hitpoints = 100
	integer replRate = 10
	
	action replanishHealth()
	output "Health has been replaneshed from " + hitpoints " to " + hitpoints+replRate
	end
	
	action attack(Hero H)
	end
	
	action replRes
	end
end

class Warrior is Hero
	int fury = 100
	
	action attack( Hero H)
	integer attackp = hitpoints - 15
	H:hitpoints = attackp
	output H + " has been slayen for " + attackp
	end
	
	action spattack( Hero H)
	integer attackp = hitpoints - 17
	H:hitpoints = attackp
	integer furyleft = fury - 10
	fury = furyleft
	output H + " has been slayen for " + attackp
	end
	
	action replRes
	fury = fury+10
	end
	
end

class Mage is Hero
	int mana = 100

	action attack( Hero H)
	integer attackp = hitpoints - 12
	H:hitpoints = attackp
	output H + " has been slayen for " + attackp
	end
	
	action spattack( Hero H)
	integer attackp = hitpoints - 15
	H:hitpoints = attackp
	integer manaleft = mana - 10
	mana = manaleft
	output H + " has been slayen for " + attackp
	end
	
	action replRes
	mana = mana+10
	end
end

end
\end{lstlisting}

\section{Participant \#7}
\begin{lstlisting}[language=Quorum,tabsize=2]
//Task 1
action Main
	
	output CalculateTotal(5, 5)
	//test the method
end

action CalculateTotal(integer numberOfOranges, integer numberOfBananas, boolean regular) returns number
	integer banana = 4
	integer orange = 5
	
	number currenTotal = 0
	
	currenTotal = (numberOfOranges mod 3) * orange + (numberOfOranges/3)*10
	currenTotal = currenTotal + (numberOfBananas mod 5) * banana + (numberOfBananas/5)*10
	
	if regular
		currenTotal = currenTotal*0.9
	end
	
	return currenTotal
end

//Task 2
use Libraries.Containers.Array
action Main
	Array<player> team1
	Array<player> team2
	

end

action SameTeamNames(Array<player> team) returns Array<player>
	Array<player> returnArray
	integer i = 0
	integer j = 1
	
	repeat while i < team:GetSize()
		repeat while j < team:GetSize()
			if team:Get(i):FirstName() = team:Get(j):FirstName() or team:Get(i):LastName() = team:Get(j):LastName()
				returnArray.Add(team:Get(i))
				returnArray.Add(team:Get(j))
			end
			j = j+1
		end
		i = i + 1
		j = i+1
	end
	
	return returnArray
end

action DiffTeamNames(Array<player> team1, Array<player> team2) return Array<player>
	Array<player> returnArray
	
	integer i = 0
	integer j = 0
	
	repeat while i < team1:GetSize()
		repeat while j < team2:GetSize()
			if team1:Get(i):FirstName() = team2:Get(j):FirstName() or team1:Get(i):LastName() = team2:Get(j):LastName()
				returnArray.Add(team1:Get(i))
				returnArray.Add(team2:Get(j))
			end
			j = j+1
		end
		i = i + 1
		j = 0
	end
	
	return returnArray
end

action SameTeamAge(Array<player> team) returns Array<player>
	Array<player> returnArray
	
	integer i = 0
	integer j = 1
	
	repeat while i < team:GetSize()
		repeat while j < team:GetSize()
			if team:Get(i):FirstName() = team:Get(j):FirstName() and team:Get(i):Age() = team:Get(j):Age()
				returnArray.Add(team:Get(i))
				returnArray.Add(team:Get(j))
			end
			j = j+1
		end
		i = i + 1
		j = i+1
	end
	
	return returnArray
end

//Task 3


//Task 4
action Main
	text t = "HenrikGeertsen"
	
	integer i  = t:GetSize()-1
	text out = ""
	repeat while i > 0
		out = out + t:GetCharacter(i)
		i = i - 1
	end
	output out
	
	Array<text> a = t:Split("")
	a:Sort()
	i = 0
	out = ""
	repeat while i < a:GetSize()
		out = out + a:Get(i)
		i = i + 1
	end
	output out
	
	i = 1
	boolean found = false
	integer duplicates = 0
	repeat while i < a:GetSize()
		if (a:Get(i) = a:Get(i-1))
			found = true
		else
			if (found)
				duplicates = duplicates + 1
			end
			found = false
		end	
	end
	output duplicates
end
\end{lstlisting}

\section{Participant \#8}
\begin{lstlisting}[language=Quorum,tabsize=2]
use Librarises.Containers.Array

class fruit 
	number _price = 0;
	
	public action RaisePrice(number newPrice) 
		me:price = newPrice
		
class banana is fruit 
	number _price = 4
	
class orange is fruit
	number _price = 5 
	
class bananas 
	Array<banana> _bananas
	public action addBanana()
		_bananas:add(banana) 
	
class oranges
	Array<orange> _oranges
	public action addOrange()
		_oranges:add(orange)
	
class calculator
	public action isRegular(bool isRegular, number price) returns number 
		if isRegular = true
			return price = price * 1.10


			
action Main 
	integer i = 0
	number OTotal = 0
	number BTotal = 0 
	
	oranges orangesLst
	bananes bananasLst
	calculator c 
	
	repeat while not(i = 10)
		orangeLst:addOrange()
		bananasLst:addBanana()
		
	repeat while i < orangeLst:GetSize()
		Ototal = OTotal + orangeLst:_oranges:Get(i):_price
		if (i mod 3) == 0
			0Total = 0Total - 5
			
	regularPrice = c:isRegular(true, OTotal)
	normal = c:isRegular(false, 0Total)
		
	repeat while i < bananasLst:GetSize()
		Ototal += bananasLst:_bananas:Get(i):_price
		if (i mod 3) == 0
			BTotal = BTotal - 10
		
	repeat while i  
	
	
	
action Main2
	int nrPlayers = 11
	
	Array<text> fullNames1
	Array<text> fullNames2
	
	fullNames1:add("martin, fruensgaard, 24")
	fullNames2:add("Tommy, something, 23")
	
	int i = 0, j = 0; 
	repeat while i < fullNames1:GetSize()
		repeat while j < fullNames2:GetSize()
			Array<text> name1 = fullNames1:Get(i):Split(",")
			Array<text> name2 = fullNames2:Get(j):Split(",")
			
			if(name1:Get(0) = name2:Get(0) or name1:Get(1) = name2:Get(1)) 
				output "BINGo!<3 2 players: " + fullNames1(i) + " and " fullNames2(j)
				
			
	int i = 0, j = 0; 
	repeat while i < fullNames1:GetSize()
		repeat while j < fullNames2:GetSize()
			Array<text> name1 = fullNames1:Get(i):Split(",")
			Array<text> name2 = fullNames1:Get(j):Split(",")
			
			if not(name1 = name2)
				if(name1:Get(0) = name2:Get(0) or name1:Get(1) = name2:Get(1))
				
	int i = 0, j = 0; 
	repeat while i < fullNames1:GetSize()
		repeat while j < fullNames2:GetSize()
			Array<text> name1 = fullNames1:Get(i):Split(",")
			Array<text> name2 = fullNames1:Get(j):Split(",")
			
			if not(name1 = name2)
				if(name1:Get(0) = name2:Get(0) and  name1:Get(2) = name2:Get(2))
\end{lstlisting}