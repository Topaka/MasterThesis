\chapter{Interview notes}
As previously mentioned, the interviews with the participants were recorded in order to preserve the necessary feedback from them when analysing the results. Instead of providing the entirely of the interviews in the form of transcripts, we decided to condense the information in key points instead. This helped us to analyse the data from the interviews much easier and find out how many occurrences of a given problem there are across all the participants. Additionally, it encapsulates the essential parts of each interview and highlights what every participant had to say in terms of feedback, suggestions for further improvements and encountered problems.
\section{Participant {\#}1 (Pilot Test)}
\begin{itemize}
\item Thought that using colon (“:”) instead of dot (“.”) was weird, both because it goes against the norm (the participant had experience with several languages which use the dot notation) and because dot is easier to type. 
\item Thought that tasks are trivial to understand but take time to code
\item Task 1 was too broad in the definition causing the task to be too large and time-consuming and the participant to spend time on unintended things.
\item Mentioned that although repetitive to a certain extend, task 2 was tricky and very good at conveying that you have to pay attention when copying code. 
(*He actually fell into that trap and he did not realise it up until the facilitator intervened and pointed that out.) 
\item Thought that tasks 2 and 4 were quite good in terms of their intended purposes while task 3 (operation on strings) was trivial and very similar to task 2. 
\item He found the samples of conditional use not being able to clearly convey the differences between Quorum and other known languages he had experience in. In particular "==" vs. "=" and "and" vs "\&\&" did not stand out.
\item Thought that it was good that the sample sheet was split in categories to make it easier for the facilitator to reference them when asked.
\end{itemize}
\section{Participant {\#}2}
\begin{itemize}
\item Thought that a lot of the notations were unintuitive because they differed from the mathematical norm
\item Found the keyword "action" confusing 
\item Thought that using "loop" would be easier than "repeat"
\item It was difficult for him to devise the code needed for solving the tasks, although he found the mathematics behind quite easy
\item It was daunting to not have any fallback or assistance when trying to code and learn how to code without a compiler
\end{itemize}
\section{Participant {\#}3}
\begin{itemize}
\item Suggested that we should add specific values for task 3
\item Wondered how to define return types of an action
\item Quorum does not have parameterized constructors
\item Suggested that we add how to get the size of an array with an in-build action
\item Forgot to increment loop counters
\item Forgot to add the “repeat” keyword
\item Thought that Quorum has a limited number of looping constructs, but it is easy to learn, write and understand
\item Quorum is very terse
\item Thought that “output” makes more sense than using “print”
\item Thought that “returns” of an action seems intuitive
\item Liked the “is” keyword for class inheritance
\item Thought that Quorum is similar to C, with a different syntax (programmer-friendly C) 
\item Thought that the lack of constructors is not that limiting, but does not have enough experience to say with a certainty
\item Found the tasks not too challenging 
\item Thought that not using a compiler is not much of a hindrance
\item Found the example sheet indispensable and very helpful
\item Suggested that we could add more examples for loping constructs
\end{itemize}
\section{Participant {\#}4}
\begin{itemize}
\item Found it strange to use words as a means of closing scopes instead of brackets as well as using colons instead of dots
\item Thought the languages is straightforward and easy to use
\item Used a “float” instead of a “number” keyword, as well as “string” instead of “text”
\item Forgot to add “returns” keyword at the end of an action
\item Forgot to increment the counters on loops
\item Had some problems with scoping by making use of the “end” keyword
\item Found the tasks specific, understandable and clear
\item Thought that “.” makes more sense than “:”
\item Suggested that we add an example of method inheritance on the example sheet
\item Suggested that we change the “/” on task 2 with an “or”
\item Suggested that we add a sort action on the example sheet
 constructs
\end{itemize}
\section{Participant {\#}5}
\begin{itemize}
\item Found Quorum is similar to C
\item Quorum has similar design to other languages “string” instead of “text”
\item Tended to over-complicate things and thus - over-engineer the tasks
\item Made use of the example sheet quite frequently
\item Coding without a compiler was unpleasant and felt like being in an exam, unable to get a feedback from what’s being written down (Does not allow a great deal of experimentation)
\item Had difficulties with the syntax of arrays - using the “[]” notation instead of the get(i) inbuilt method
\item Forgot to write the import for using arrays, as specified on the example sheet
\item Found the tasks very good at conveying our intended purposes and easy to understand
\item Found the amount of tasks good and reasonable
\item Found Task 3 to be a bit tricky since you have to specifically think in terms of inheritance from the start
\item Found the example sheet informative and referred to it several times
\end{itemize}
\section{Participant {\#}6}
\begin{itemize}
\item Quorum’s design  seems a bit confusing
	\begin{itemize}
		\item Closing the scopes of If-statements with “end”
		\item Lack of parameterized constructors
		\item Lack of a for-loop
	\end{itemize}
\item Found the tasks very good and the example sheet - very concise
\item Found coding without a compiler scary without “the safety net”
\item Typed “=” instead of “==” for an inequality operator
\item Thought it might be more intuitive to use a Get method directly compared to how it is being used in the language
\item Typed “.” instead of “:”
\item Thought that ending classes with something different than the “end” keyword will make more sense
\item Found closing the scope of if-statements with the “end” keyword confusing and said that brackets would make it more readable (similarly to OO languages such as Java and C\#)
\item Found Quorum less verbose than other OO languages
\item Forgot to increment the counter variable outside of a loop
\item Although the participant over complicated the tasks based on the provided description, he found them very good and efficient at what they try to convey
	\begin{itemize}
		\item Task 2 - the description of the task seems rather confusing, which made the participant to over-engineer the solution
		\item Task 3 - doable 
	\end{itemize}
\item Found the example sheet contains enough content in order to solve the tasks  
\item Had a few suggestions how to improve the overall look of the example sheet
\end{itemize}
\section{Participant {\#}7}
\begin{itemize}
\item Found Quorum similar to Pascal and C\#
\item Liked certain parts of the language and disliked others
\item Found the use of “repeat” unnecessary since it does not make sense in conjunction with the standard loop wording
\item Noticed that you have to close a class/action with an “end” keyword
\item Suggested that implicit type casting would be better for novices
\item “:” used in different scenarios might be confusing
\item Thought that the “returns” keyword can have a better placement in the action’s signature
\item Noticed that you have to use a library for an array
\item Said that the “end” keyword does not make much sense and rather see a “begin-end” scoping construct, similar to Pascal and Python - only indentation
\item Casting data types could be dangerous for novices
\item Found the “returns” keyword’s placement not so intuitive 
\item Found the “end” keyword for if-statements not so adequate, can use indentation instead similar to Python
\item Found the tasks very good:
	\begin{itemize}
		\item Task 2’s challenge of reusing code is a good exercise 
		\item Task 2 could have a 2 predefined lists with names
	\end{itemize}
\item Said that the task encompass a good portion of constructs 
\item Suggested we could add a setup for easier start with the tasks
\item Suggested we give better titles on the examples sheet and better indexing when looking for things
\item Coding without a compiler did not matter that much in his opinion
\item Found it great that the facilitator could say if the task is done or not
\item GetSize() and Add() in-build methods examples were missing
\item Acknowledged that the code samples are highlighted and there are working examples
\item Said that we should be consistent with the working titles
\end{itemize}
\section{Participant {\#}8}
\begin{itemize}
\item Found Quorum intuitive to use, but limited in terms of available constructs
\item Suggested that “returns nothing” would be intuitive
\item Found the naming of keywords inconsistent (Arrays with capital A and everything else with small letters)
\item Found it confusing not to use indentation for scopes
\item Found the lack of semicolons a very good thing
\item Liked the “is” keyword for class inheritance
\item Pointed out the lack of a “continue” construct for loops
\item Would have liked more features from functional programming
\item Suggested we could make the “or” and “and” statements bolded in task 2
\item Noticed the lack of an aggregate “+= “ operator
\item Quorum reminds him of OO languages and similar to Python
\item Would have liked a summary of all the examples on the examples sheet
\item Found the examples not so sufficient per task
\item Suggested that we could highlight important parts on the task sheet 
\item Found the lack of a compiler while coding “dangerously scary”
\item Over-engineered task 1 
\item Suggested that we could have an additional sheet with solutions to the tasks
\item Separate each task on a separate paper so it is easier to navigate 
\end{itemize}
\chapter{afdsfa}
