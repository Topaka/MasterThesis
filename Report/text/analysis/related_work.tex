\chapter{Related Work}
\label{chap:related_work}

Programming languages have been used for years but still there has not been established a robust and efficient way to asses and evaluate them. However, plenty of research has been done on the topic and specific papers address that to a different degree as it will be shown in this chapter.

M. Farooq et al. 2014 \todo{ref needed}, wrote a paper introducing an evaluation framework which provides a comparative analysis of widely used first programming languages (FPLs), or namely languages which are used as a first language for teaching introductory programming. The framework is based on technical and environmental features and it is used for testing the suitability of existing imperative and object oriented languages as appropriate FPLs. In order to support their framework, they have devised a customizable scoring function for computing a  suitability score for a given language which helps quantify and rank languages based on the given criterion. Lastly, they evaluated the most widely used FPLs by computing their suitability scores. The novelty in their work stems from the definition of the evaluation parameters as well as the related characteristics to evaluate each parameter.

K. Parker et al. 2006 \todo{ref needed} note that the process of selecting a programming language for introductory courses is informal and it lacks structure and replicability. In order to address that, a more structured approach is proposed which enables a more detailed evaluation of the selection process. The paper presents an exhaustive list of selection criteria  where each criterion is assigned weights in order to determine its relative importance in the selection process. This is tested by scoring different programming languages according to the said criteria. The proposed approach is verified by an informal pilot study to assess the completeness of the criteria and gather enough feedback on the process itself. Given the dynamic nature of programming paradigms and languages, the authors acknowledge that the selection criteria and process can be revised.








   