\chapter{Challenges}
\label{chap:challenges}

Conducting empirical studies in order to prove the validity of a hypothesis usually involves a certain amount of human resources. For studies and experiments having a qualitative nature this number is in the tens or hundreds, and for ones having a quantitative nature - in the thousands. Gathering data, even from a small number of people is a daunting task and takes considerable resources and time. Additionally, the problem arises of trying to have a participants sample with diverse backgrounds (age, occupation, experience etc.) in order to address a wider population, which adds an additional layer of complexity. 

Although the qualitative nature of our experiment does not require a large number of participants to show viable results, we still had difficulties in gathering the necessary number of participants. Some  reasons for this might be that people do not find a good enough incentive to take part in such experiment, or they simply find the process intimidating, time consuming or not important. We found that approaching people directly rather than through electronic means (email, forums, on-line conversations) yields a higher chance of them wanting to take part as participants. 
