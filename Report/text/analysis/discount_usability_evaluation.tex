\section{Discount usability evaluation method}
When we are talking about the discount evaluation method, we are talking about the cooperative evaluation created by Andrew Monk et al. 1993\cite{AndrewMonk}. Table \ref{fig:UsabilityTable} from Designing Interactive Systems\cite{CooperativeEval} outlines the steps from the method.

\begin{figure}[]
  \centering      
    \begin{subfigure}[b]{\textwidth}
    \begin{center}
      \includegraphics[scale=0.5]{./pics/UsabilityTableP1}
      \caption{table part 1}
      \label{fig:UsabilityTableP1}
    \end{center}
    \end{subfigure}
    ~\\
    \begin{subfigure}[b]{\textwidth}
    \begin{center}
      \includegraphics[scale=0.5]{./pics/UsabilityTableP2}
      \caption{table part 2}
      \label{fig:UsabilityTableP2}
    \end{center}
    \end{subfigure}
    \caption{combined caption}
    \label{fig:UsabilityTable}
\end{figure}

This method is focused on finding the bigger usability problems over finding a lot of them, and has thus proven to be both lightweight and useful. In particular the Nielsen Norman Group\cite{5Users} shows that you only need about 5 test subjects to get clear feedback on your system. Jakob Nielsen argues that instead of expending huge budget and time, the best results could be achieved by conducting multiple smaller tests. Across a large number of projects, one test subject managed to find roughly 31\% of the the existing usability problems. Every subsequent user contributes to the curve by identifying new problems, but that also includes problems already found by the previous user. Technically, after a certain point, adding additional users contribute less and less to the overall identification of problems, having essentially a diminishing returns effect. The curve presented by the author shows that 15 test users are needed to identify all the usability problems, but rather than conducting one experiment with all 15 users, a much better solution is to conduct 3 smaller tests with 5 users each. The main reason for that is improving the design not just documenting its weaknesses.