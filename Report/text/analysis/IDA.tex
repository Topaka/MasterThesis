\chapter{Instant Data Analysis}
\label{chap:IDA}
After conducting a usability study the data needs to be analysed and formulated into a list of usability problems.
There are multiple ways of doing this, but the one we will be highlighting is the Instant Data Analysis (IDA)\cite{IDA} method.
The IDA method works by identifying a list of problems, that participants encountered during the test, from the observers' memories and observations.
This is opposed to the Video Data Analysis (VDA) method, where the videos of the tests are rigorously examined to identify the usability problems.
The problems are then sorted into three categories: cosmetic, serious and critical.
\begin{description}
\item[Cosmetic problems] briefly slowed down the participant but were quickly overcome.
\item[Serious problems] stumped the participant for longer time, but were eventually overcome.
\item[Critical problems] the participant could not overcome without assistance from the facilitator.
\end{description}
This prioritisation then usually becomes a great assistance in determining which problems are the most important to fix.

The IDA method is a good analysis method for our purposes because it is lightweight.
Compared to the VDA method, the IDA method saves a lot of time.
Avoiding the rigorous analysis of the video footage saves a lot of man-hours, as shown in the paper\cite{IDA} where they spent 4 man-hours with the IDA method as opposed to 40 man-hours with the VDA method.
Furthermore the time to result is greatly reduced since the method uses these man-hours in a joint discussion causing the real time spent to be around 1-2 hours.%this sentence is own observation and might not be relevant
As for the results, In the IDA paper\cite{IDA} the IDA method found 85\% of the same critical problems as the VDA method and found roughly the same amount of problems total (41 for IDA versus 46 for VDA).
%maybe a sentence highlighting the different errors found rather than no error found
This makes the IDA method a great method for our purposes since the potential loss problem accuracy is significantly lower than the time saved.