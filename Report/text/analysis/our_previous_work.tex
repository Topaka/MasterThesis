\chapter{Previous work}

In our previous work \cite{SemesterProject}, we explored how computational thinking becomes a necessary skill in everyday life and how introductory programming has been adopted by several countries (such as UK, USA, Denmark, Germany and others) as a part of their primary school curriculum. A big part of this process involves the choice of a suitable programming language and environment and given the multitude of choices available, there has not been a well established way of selecting the right choice. Furthermore, there exist different approaches to programming (visual-based vs. text-based), each with their own strengths and weaknesses, which makes that choice even harder. In order to get a better understanding of how this choice could be made easier, we analysed three popular programming languages used in a educational setting - Java (BlueJ), Scratch (Scratch) and Racket (DrRacket), each supporting a different programming paradigm. The analysis was done mainly relying on a set of evaluation criteria, heavily inspired by Sebesta's evaluation criteria, on a set of programming tasks in the form of games. Although this approach gave us some pointers towards how effective and
suitable each of tested languages is, the evaluation was rather subjective and did not possess the needed scientific rigour and empirical data appertaining for such a method. This however prompted us to look into how techniques from social sciences \cite{Bryman} could be used to gather the necessary data, in particular the discount usability evaluation method by A.Monk \cite{AndrewMonk}.