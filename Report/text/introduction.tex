\chapter{Introduction}
\label{chap:introduction}

Computer programming has increasing relevance to today's advancement of technologies. Therefore, existing and established programming languages are constantly improved and new ones are created to meet that demand. The languages which are considered most suitable for introductory programming, are being adopted by educational institutions as part of their computer science curriculum e.g. Java, Python and more recently, Scratch\cite{PracticeTeachingIntro}. Similarly, some languages are considered arguably better than others in their intended purpose in the software industry. However, formal evaluation methods for assessing programming languages are very few and limited in their use and most evidence gathered to support such claims are anecdotal in nature\cite{StakingClaims}. 

In recent years, however, the scientific community has tried to fix this.
In particular, the focus of the PLATEU conferences is the scientific evaluation of languages.
The basic observation is that language use and preference is highly dependant on the person, which indicates that there is no method that can tell the quality of the language based solely on the language.
This has lead to user-based evaluation being the norm for programming languages.
Commonly, the scientific community has taken to use methods from social sciences, which usually requires a studying a large number of subjects\cite{SocioPLT}\cite{AliceCS1}\cite{BlockOrNot}\cite{FromScratch}.
The underlying idea is that the more people used, the less likely the result would suffer from bias, and therefore the study will be more scientific.

However, his method has the problem of being expensive.
It is difficult to gather a large enough group of people to create a truly quantitative test of a language, especially if the test is based on observation rather than questionnaires.
Typically this means a language designer cannot do these tests before the language is already finished and has gained widespread use.
This leads to language designers either still omitting any evaluation of their language, or using more qualitative and lightweight approaches to evaluate their language.

In the field of Human-Computer Interaction (HCI) there has been a similar problem with usability evaluation methods \cite{IDA}.
Traditional usability studies often require many subjects and studying the use of the program over a long period of time.
This made them prohibitively costly for companies both in resources spent and time-to-market, which prompted the development of a discount usability evaluation method\cite{AndrewMonk}.
The discount usability evaluation method is a method designed to be a lightweight, qualitative evaluation method for the usability of a product.
We are considering its viability as an evaluation method for programming languages.

\section{Problem formulation}
\label{section:problem formulation}
During our investigation, we concluded that there is gap between internal language analysis and the big and bulky industry programming evaluation methods\cite{AliceCS1}\cite{BlockOrNot}\cite{FromScratch}, which leaves language designers without a flexible, low-cost and efficient solution to test on new programming languages. We want to have a method for evaluating programming languages, which is lightweight enough to be used by language designers, both for improving the language, and supporting claims about the language. The discount usability method is a method that is lightweight enough to be used by small groups of system designers and seems like it could be applicable in that regard. Ideally, we would use that as a starting stage for devising our own lightweight evaluation method which could be used in different stages of programming language design. This leads us to the following initial questions:

\subsection{Initial questions}
\begin{itemize}
\item Are the evaluation techniques from HCI applicable to programming languages evaluation?
	\begin{itemize}
		\item What does the usability of programming languages say about programming language design?
	\end{itemize}
\item Can the discount usability method be used for programming language evaluation?
	\begin{itemize}
		\item What are the potential shortcomings of the method for evaluating programming languages?
		\item How can such shortcomings be addressed?
	\end{itemize}
\end{itemize}
