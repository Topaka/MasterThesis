\chapter{Introduction}
\label{chap:introduction}

Computer programming has increasing relevance to today's advancement of technologies. Therefore, existing and established programming languages are constantly improved and new ones are created to meet that demand. The languages which are considered most suitable for introductory programming, are being adopted by educational institutions as part of their computer science curriculum e.g. Java, Python and more recently, Scratch \todo{Maybe a reference(s) is needed here} . Similarly, some languages are considered arguably better than others in their intended purpose in the software industry. However, formal evaluation methods for assessing programming languages are very few and limited in their use and most evidence gathered to support such claims are anecdotal in nature. 

In recent years however the scientific community has tried to fix this.
In particular the focus of the PLATEU conferences is the scientific evaluation of languages.
The basic observation is that language use and preference is highly dependant on the person, which indicates that there is no method that can tell the quality of the language based solely on the language.
This has lead to user based evaluation being the norm for programming languages.
Commonly the scientific community has taken to use methods from social sciences, which usually requires a studying a large number of subjects.
The underlying idea is that the more people used the less likely the result would suffer from bias and therefore the study will be more scientific.

This method however has the problem of being expensive.
It is difficult to gather a large enough group of people to create a truly quantitative test of a language, especially if the test is based on observation rather than questionnaires.
Typically this means a language designer cannot do these tests before the language is already finished and has gained widespread use.
This leads to language designers either still omitting any evaluation of their language, or using a more qualitative and lightweight approach to evaluate their language.
The discount usability evaluation method is a method designed to be a lightweight, qualitative evaluation method for the usability of a product.
We are considering its viability as an evaluation method for programming languages.

\section{Initial Questions}
We want to have a method for evaluating programming languages, which is lightweight enough to be used by language designers, both to improve the language, and support claims about the language.
The discount usability method is a method that is lightweight enough to be used by small groups of system designers and seems like it could be applicable.
This leads us to the following initial questions:

\begin{itemize}
\item How good is the discount usability method for language evaluation.
\item How can the method be changed to create a good method for language evaluation.
\end{itemize}

\section{Motivation}