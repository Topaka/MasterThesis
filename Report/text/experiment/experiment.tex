\chapter{Experiments}
\section{Usability Experiment}
To get an understanding of how applicable the discount usability method is for evaluating programming languages, we used the IDA method\todo{add source} on C# and F#.
The usability evaluation was conducted in the AAU usability lab\{add source} in test room 1.
The subject was positioned at a desk with a mouse, keyboard and screen.
On the screen Visual Studio\todo{add specific version} was open with a project in the current language being tested.
The subject was then given an hour to try and solve some task(s) in the language.
Next to the subject a coordinater sat.
His job was primarily to keep the subject talking about what they where thinking, and secondarily to assist if the subject got stuck or encountered issues unrelated to the test.
Meanwhile an observer would take notes from an adjacent observation room.
The \todo{insert specific name for control room} was used for this.
Despite using IDA the tests where recorded.
A camera would record the subjects interaction with the keyboard and mouse, while the screen was recorded using an output splitter, and the sounds was recorded using a microphone.

For C# our task was:\todo{might be more fitting to have in appendix}
you want to create a system for managing stats and interactions between RPG characters with different classes.
Each character has:
-a name
-a character class
-a stat that represent their hit points
-one or more stats that represents offensive prowess
-one or more stats that represents defensive prowess
-optional class specific stats
-the ability to attack another character, which reduces their hit points by value related to the difference in character1's offense and character2's defense
-optional the ability to perform class specific skills

here are some examples of characters:
name 	george		bob					john
class	warrior		mage					medic
hp	200		150					200
offense	50		100					30
defense	50		10					60
			mana = 100
skills:	attack		attack					attack
					Fireball - more damage but costs mana	mending - increases the hit points of a character
\todo{fix formatting to make it pretty}

Create the system for these RPG characters with some character classes, some characters in this system and some code showing some interactions.

now we discover we also want a way to refresh characters to their original hit points, as well as preventing healing beyond this point.
To do this we need to also store a maximum hit points value for each character as well as a method for restoring a characters resources.
Some class specific resources might also need this to allow them to be replenished as well.

The task was designed for subjects who already had experience with C# and know about object-oriented programming.
It was designed to show an example of programming classes with inheritance, one of C# strengths.
The wording was intentionally left vague and specifically avoided directly mentioning programming constructs, in order to avoid biasing the solution in a particular direction, as well as more accurately show how the language was used.