\chapter{Experiments}
\section{Usability Experiment}\todo{think about consistent time tense}
To get an understanding of how applicable the discount usability method is for evaluating programming languages, we used the IDA method\todo{add source} on C\# and F\#.
The usability evaluation was conducted in the AAU usability lab \todo{add source} in test room 1.
The subject was positioned at a desk with a mouse, keyboard and screen.
On the screen Visual Studio Community 2015 was open with a project in the current language being tested.
The subject was then given an hour to try and solve some task(s) in the language.
Next to the subject a coordinater sat.
His job was primarily to keep the subject talking about what they where thinking, and secondarily to assist if the subject got stuck or encountered issues unrelated to the test.
Meanwhile an observer would take notes from an adjacent observation room.
The \todo{insert specific name for control room} was used for this.
Despite using IDA the tests where recorded.
A camera would record the subjects interaction with the keyboard and mouse, while the screen was recorded using an output splitter, and the sounds was recorded using a microphone.

\subsection{C\# Tasks}
For C\# our task was:\todo{might be more fitting to have in appendix}

you want to create a system for managing stats and interactions between RPG characters with different classes.
Each character has:
\begin{itemize}
\item a name
\item a character class
\item a stat that represent their hit points
\item one or more stats that represents offensive prowess
\item one or more stats that represents defensive prowess
\item optional class specific stats
\item the ability to attack another character, which reduces their hit points by value related to the difference in character1's offense and character2's defense
\item optional the ability to perform class specific skills
\end{itemize}

here are some examples of characters:
\begin{table}[h]
\centering
\label{my-label}
\begin{tabular}{llll}
Name        & George & Bob                                   & John                                      \\
Class       & None   & Mage                                  & Medic                                     \\
HP          & 200    & 100                                   & 200                                       \\
Offense     & 50     & 60                                    & 20                                        \\
Defense     & 30     & 10                                    & 60                                        \\
Other stats &        & mana = 100                            &                                           \\
Abilities   & Attack & Attack                                & Attack                                    \\
            &        & Fireball = more damage but costs mana & Mending = increases HP of target          \\
            &        &                                       & character
\end{tabular}
\end{table}

Create the system for these RPG characters with some character classes, some characters in this system and some code showing some interactions.

now we discover we also want a way to refresh characters to their original hit points, as well as preventing healing beyond this point.
To do this we need to also store a maximum hit points value for each character as well as a method for restoring a characters resources.
Some class specific resources might also need this to allow them to be replenished as well.
\\\\
The task was designed for subjects who already had experience with C\# and know about object-oriented programming.
It was designed to show an example of programming classes with inheritance, one of C\# strengths.
The wording was intentionally left vague and specifically avoided directly mentioning programming constructs, in order to avoid biasing the solution in a particular direction, as well as more accurately show how the language was used.

\subsection{C\# Results}
In our experiments on C\# we got two subjects. This is less than the desired minimum of five subjects for a usability test. However since the focus of our experiments is on the method and not on the results of the experiment it is probably not a big problem. The results of the experiment is evaluated using the IDA method, and therefore focused on a discussion based on the observations made. The results of the discussion is seen in table \ref{C-usability-results}.

\begin{table} []
\caption{The results of the C\# experiment}
\centering
\renewcommand{\arraystretch}{1.5}
\label{C-usability-results}
\begin{tabular}{| p{5cm} | p{5cm} | p{5cm} |}
\hline
Critical & Serious     & Cosmetic \\ \hline
		 & Inheritance: the subject experienced difficulties identifying when inheritance could be useful & Could not invoke the auto generator for getters and setters \\ \hline 
		 & & Difficulties remembering the keyword for declaring a method override-able (Virtual) \\ \hline
		 & & Troubles remembering the special syntax for invoking the base constructor \\ \hline
\end{tabular}
\end{table}

Most of these issues were only experienced by one of the subjects, and usually reflected their experience.
The 4th semester student was the one experiencing the serious issue due to not having had much need for inheritance in any of his previous projects.
On contrast he did not have any difficulties remembering keywords due to recent use of the language.
This is in contrast with the 10th semester student, who had more experience using inheritance but had not used C\# as recently and therefore had difficulties regarding specific keywords.
Neither of them invoked the auto generator for getters and setters and instead chose to make the variables public instead. They did however both express knowledge that this was not ideal and that it was done due to laziness and the small scope of the project.