\chapter{Future works}
\label{chap:further_works}
In this chapter a discussion of things that can be done in the future will be provided.
\\
One of the things that could be done would be to conduct the usability experiment on Quorum still with experienced programmers using the sodbeans environment.
This would give a more direct comparison of the differences in the data gotten from using the usability experiment method versus our method.

Of course simply conducting our method on more languages, and ideally by the language designers of these languages, would also give a lot of data about the method.
One way to facilitate this could be to spread the method to the 4th semester students in software and computer science at Aalborg University.
Since these student have to design a language as part of their semester project anyway and could use the data to argument for their language design decisions, it would be an opportune way of testing the method in a low-risk environment.

Another thing worth exploring is using our method on novices.
Using experienced programmers makes it easier to convey how to program in a language, as they already know how to program, and it  makes sense when programmers are the target group for the language.
I does however mean that the data tend to be biased towards the languages the programmers already know.
Using novices avoids this bias and is obviously useful for languages designed for them.
It does however present a challenge for our method as novices are less familiar with the act of programming and are more prone the feel lost without the assistance of an IDE.
One example of this is participant \#2 who we have otherwise omitted from the results since he was not an experienced programmer.
In this test the participant was completely unable to write anything before the facilitator stepped in and essentially dictated exactly what should be written for the first subtask.
After this the participant was however capable of solving the second and third subtask on his own.
This shows that it is definitely not impossible to test on novices, but it probably requires some additional thought put into the task and sample sheet.
Conducting the experiment on novices would give us a better idea of what kind of alterations the experiment setup would need to facilitate those experiments.

something about testing on a more diverse group

using a bigger sample group to get a quantitative experiment

talk about conducting the experiment with a pre-made skeleton for the tasks

using a silent compiler in the experiment to allow the participant to get feedback, but without getting assistance from the tool.